\section{点集拓扑}

\subsection{基础定义}

\subsubsection{开集}

拓扑是用来度量连续性的.

\begin{definition}[拓扑空间]
    拓扑空间 (topological space) 指资料 \((X,\mathcal{T})\) 使得 \(X\) 是集合且 \(\mathcal{T} \subseteq \mathcal{P}(X)\), 并满足:
    \begin{enumerate}
        \item \(\varnothing, X \in \mathcal{T}\)
        \item 有限交 (finite intersection) \(\forall A_1, \dots, A_n \in \mathcal{T} (\bigcap_{i=1}^n A_i \in \mathcal{T})\)
        \item 任意并 (arbitrary union) \(\forall \mathcal{A} \subseteq \mathcal{T} (\bigcup_{A \in \mathcal{A}} A \in \mathcal{T})\)
    \end{enumerate}
    称 \(\mathcal{T}\) 中的元素为开集, 闭集定义为开集 \(U\) 的补集 \(X \setminus U\), 有时亦只用 \(X\) 代拓扑空间.
\end{definition}

\begin{definition}
    对于任何集合 \(X\), 分别取 \(\mathcal{T} = \mathcal{P} (X)\) 与 \(\mathcal{T} = \{\varnothing,X\}\) 可得到两个拓扑
    分别称离散拓扑 (discrete topology) 与凝聚拓扑 (indiscrete topology).
\end{definition}

\begin{example}[Sierpinski 二点集]
    \setlabel {Sierpinski 二点集}
    \label {example:sierpinski two point set}
    取 \(\mathcal{T} = \{\varnothing, \{x\}, \{x,y\}\}\) 可得到 Sierpinski 二点集.
\end{example}

\begin{example}
    在任何集合上可以定义余有限拓扑 (cofinite topology), 即取 \(\mathcal{T} = \{S \subseteq X : \abs{X \setminus S} < \aleph_0\}\).
\end{example}

\begin{definition}
    假定 \(X\) 上有两个拓扑 \(\mathcal{T}\) 与 \(\mathcal{T}^\prime\), 如果 \(\mathcal{T} \subseteq \mathcal{T}^\prime\), 则称 \(\mathcal{T}^\prime\) 比 \(\mathcal{T}\) 细 (finer),
    反之如果 \(\mathcal{T}^\prime \subseteq \mathcal{T}\), 则称 \(\mathcal{T}^\prime\) 比 \(\mathcal{T}\) 粗 (coarser).
\end{definition}

\begin{lemma}
    对 \(X\) 上拓扑 \({(\mathcal{T}_i)}_{i \in I}\) 的交 \(\bigcap_{i \in I} \mathcal{T}_i\) 是拓扑.
\end{lemma}

\begin{definition}[拓扑基]
    给定 \(X\) 与 \(\mathcal{B} \subseteq \mathcal{P}(X)\), 如果 \(\mathcal{B}\) 满足:
    \begin{enumerate}
        \item \(\forall x \in X \exists B \in \mathcal{B} (x \in B)\)
        \item 对任意 \(x,A,B\) 假使 \(x \in A \cap B\) 且 \(A,B \in \mathcal{B}\), 则存在 \(C \in \mathcal{B}\) 使得 \(x \in C \subseteq A \cap B\)
    \end{enumerate}
    则称 \(\mathcal{B}\) 为 \(X\) 上的拓扑基 (topological base).
\end{definition}

\begin{definition}
    如果 \(\mathcal{T}\) 是包含 \(\mathcal{B}\) 的最粗拓扑, 则称 \(\mathcal{B}\) 为 \(\mathcal{T}\) 的拓扑基, 称 \(\mathcal{T}\) 由 \(\mathcal{B}\) 生成.
\end{definition}

\begin{lemma}
    \label {lemma:topology generated by base}
    一个集合 \(U \subseteq X\) 在 \(\mathcal{B}\) 生成的拓扑中开当且仅当 \(\forall x \in U : \exists B_x \in \mathcal{B} (x \in B_x \subseteq U)\).

    \begin{proof}
        这些集合 \(U\) 总是开, 因为 \(U = \bigcup_{x \in U} B_x\).

        这些集合构成拓扑, 只需逐条验证:
        \begin{enumerate}
            \item \(\neg (\exists x \in \varnothing)\) 且 \(\forall x \exists B \in \mathcal{B} (x \in B)\), 故 \(\varnothing, X\) 开.
            \item 任意给出集合 \(U,V\) 开, 则只需注意到 \(x \in B_x \subseteq B_x^U \cap B_x^V \subseteq U \cap V\), 故 \(U \cap V\) 仍开, 有限交无非是二元情况下的延伸.
            \item 任意给出 \({(U_i)}_{i \in I}\), 只需取 \(B_x\) 为某个 \(U_i\) 中 \(x\) 对应的 \(B_x\) 即可.
        \end{enumerate}
    \end{proof}
\end{lemma}

\begin{definition}[邻域]
    对 \(x \in X\), \(x\) 的邻域是指集合 \(S \subseteq X\), 使得有开集 \(U\), \(x \in U \subseteq S\),
    全体 \(x\) 的邻域构成的集合记为 \(\mathcal{T}_x\).
\end{definition}

\begin{definition}[闭包]
    对 \(A \subseteq X\), \(A\) 的闭包 (closure) 是指 \(\bigcap_{\substack{A \subseteq U \\ U \in \mathcal{T}}} U\), 记作 \(\overline{A}\).
\end{definition}

闭包是包含 \(A\) 的最小闭集.

\begin{definition}[内部]
    对 \(A \subseteq X\), \(A\) 的内部 (interior) 是指 \(\bigcup_{\substack{U \subseteq A \\ U \in \mathcal{T}}} U\), 记作 \(\mathring{A}\).
\end{definition}

内部是被 \(A\) 包含的最大开集.

\begin{corollary}
    \(X \setminus \mathring{A} = \overline{X \setminus A}\).
\end{corollary}

\begin{definition}[边界]
    对 \(A \subseteq X\), \(A\) 的边界 (boundary) 是指 \(\overline{A} \setminus \mathring{A}\), 记作 \(\partial A\).
\end{definition}

\begin{definition}
    \label {definition:topological space's category}
    对于拓扑空间 \((X,\mathcal{T})\), 定义其对应的范畴 \(\mathrm{Cat} (\mathcal{T})\) 如下:

    \begin{enumerate}
        \item 对象是 \(\mathcal{T}\) 中的元素.
        \item \(X \subseteq Y\) 时有唯一态射 \(X \to Y\).
        \item \(X \nsubseteq Y\) 时没有态射.
    \end{enumerate}
\end{definition}

\begin{lemma}
    拓扑空间对应的范畴有有限积与任意余积.
\end{lemma}

\begin{definition}
    对于拓扑空间 \((X,\mathcal{T}_x), (Y,\mathcal{T}_y)\) 与映射 \(f : X \to Y\),
    如果 \(\forall U \in \mathcal{T}_Y (f^{-1} (U) \in \mathcal{T}_X)\), 则称 \(f\) 连续 (continuous).
\end{definition}

\begin{lemma}
    连续映射的复合仍然连续.

    \begin{proof}
        对于连续映射 \(f : X \to Y, g : Y \to Z\), 有 \(\forall U \in \mathcal{T}_Z (g^{-1} (U) \in \mathcal{T}_Y)\) 且 \(\forall V \in \mathcal{T}_Y (f^{-1} (V) \in \mathcal{T}_X)\),
    \end{proof}
\end{lemma}

\begin{corollary}
    连续映射 \(f : X \to Y\) 诱导出函子 \(\mathrm{Cat} (\mathcal{T}_y) \to \mathrm{Cat} (\mathcal{T}_x)\).
\end{corollary}

\begin{definition}
    定义拓扑空间范畴 \(\mathbf{Top}\) 如下:
    \begin{enumerate}
        \item 对象是拓扑空间.
        \item \(\mathrm{Hom}_{\mathbf{Top}} (X,Y)\) 是所有连续映射 \(f : X \to Y\).
    \end{enumerate}
\end{definition}

\begin{corollary}
    上述定义的 \(\mathbf{Top}\) 有显见的初对象 \(({\bullet},\{\varnothing, \{\bullet\}\})\) 与终对象
    \((\varnothing,\{\varnothing\})\).
\end{corollary}

\begin{corollary}
    上述定义的 \(\mathrm{Cat}\) 给出 \(\mathbf{Top} \to \mathbf{Cat}\) 的一个反变函子.
\end{corollary}

\begin{definition}
    对拓扑空间范畴 \(\mathbf{Top}\) 定义遗忘函子 \(U : \mathbf{Top} \to \mathbf{Set}\) 如下:

    \begin{enumerate}
        \item 映拓扑空间 \((X,\mathcal{T})\) 到集合 \(X\).
        \item 映连续映射 \(f : (X,\mathcal{T}_x) \to (Y,\mathcal{T}_y)\) 到 \(f : X \to Y\).
    \end{enumerate}
\end{definition}

\begin{lemma}
    遗忘函子有自然的左伴随函子 \(\mathbf{Set} \to \mathbf{Top}\), 赋予每个集合离散拓扑.

    同理, 遗忘函子有自然的右伴随函子 \(\mathbf{Set} \to \mathbf{Top}\), 赋予每个集合凝聚拓扑.
\end{lemma}

\begin{definition}[子空间拓扑]
    给出单射 \(f : Y \to U X\), \(Y\) 上可以赋予最粗的拓扑使得 \(f\) 连续, 即 \(\mathcal{T}_Y = \{f^{-1} (U) : U \in \mathcal{T}_X\}\).

    当 \(Y \subseteq X\) 时, 可以取 \(f : Y \to U X\) 为包含映射, 上述拓扑称为子空间拓扑.
\end{definition}

\begin{definition}[商拓扑]
    给出满射 \(g : U X \to Y\), \(Y\) 上可以赋予最细的拓扑使得 \(g\) 连续, 即 \(\mathcal{T}_Y = \{U \subseteq Y : g^{-1} (U) \in \mathcal{T}_X\}\).

    当给出 \(X\) 上等价关系 \(R\), \(Y = X/R\) 时, 上述拓扑称商拓扑.
\end{definition}

\begin{lemma}
    上述所论两个拓扑满足如下性质 (仍取 \(f,g\)):

    对于任何 \(\phi : Z \to Y\), \(\phi\) 连续当且仅当 \(f \circ \phi\) 连续.
    对于任何 \(\phi : Y \to Z\), \(\phi\) 连续当且仅当 \(\phi \circ g\) 连续.
\end{lemma}

\begin{definition}[同胚]
    \(\mathbf{Top}\) 中的同构称为同胚 (homeomorphism).
\end{definition}

\begin{corollary}
    同胚给出拓扑空间的双向连续双射.
\end{corollary}

\begin{definition}[积拓扑]
    在 \(\mathbf{Top}\) 中, 一族拓扑空间的直积上的拓扑称积拓扑.

    \begin{proof}
        需证明这样的拓扑空间与拓扑存在.

        对于一族拓扑空间 \({(X_i, \mathcal{T}_i)}_{i \in I}\), 可以给出笛卡尔积 \(X = \prod_{i \in I} X_i\) 与投影映射
        \(\pi_j : X \to X_j\), 令 \(\mathcal{T}\) 为包含所有 \(\pi_j^{-1} (U)\) 的最粗拓扑, 则 \((X,\mathcal{T})\) 给出 \(\mathbf{Top}\) 中直积.
    \end{proof}
\end{definition}

\begin{corollary}
    积拓扑有拓扑基 \(\{\bigcap_{i = 1}^n \pi_i^{-1} (U_i)\}\).
\end{corollary}

\begin{definition}[余积拓扑]
    同理, 定义 \(\mathbf{Top}\) 中, 余积拓扑为一族拓扑空间的余积上的拓扑.

    \begin{proof}
        给 \(\mathbf{Top}\) 中一族拓扑空间 \({(X_i, \mathcal{T}_i)}_{i \in I}\), 可以给出不交并 \(X = \coprod_{i \in I} X_i\) 与包含映射
        \(\iota_j : X_j \to X\), 令 \(\mathcal{T}\) 为包含所有 \(\iota_j^{-1} (U)\) 的最细拓扑, 则 \((X,\mathcal{T})\) 给出 \(\mathbf{Top}\) 中余积.
    \end{proof}
\end{definition}

\begin{corollary}
    一个 \(U = \coprod_{i \in I} U_i\) 在余积拓扑中开当且仅当 \(\forall i \in I (U_i \in \mathcal{T}_i)\).
\end{corollary}

\begin{definition}[连通]
    对于拓扑空间 \((X,\mathcal{T})\), 如果其不能被表示为两个非空拓扑空间的不交并, 则称其连通 (connected).
\end{definition}

\begin{lemma}
    拓扑空间连通当且仅当没有既开又闭的非平凡子集.
\end{lemma}

\begin{lemma}
    \(\mathbf{Top}\) 中有任意极限与余极限, 因为其有积与余积, 核与余核.
\end{lemma}

\begin{axiom}[Kuratowski 闭包公理]
    \setlabel {Kuratowski 闭包公理}
    \label {axiom:kuratowski closure}
    闭包运算满足以下性质, 且任意给出闭包映射 \(\mathcal{P} (A) \to \mathcal{P} (A)\) 满足以下性质, 亦给出 \(X\) 上的唯一拓扑:

    \begin{enumerate}
        \item \(\overline{\varnothing} = \varnothing\)
        \item \(A \subseteq \overline{A}\)
        \item \(\overline{\overline{A}} = \overline{A}\)
        \item \(\overline{A \cup B} = \overline{A} \cup \overline{B}\)
    \end{enumerate}

    \begin{proof}
        定义 \(\overline{C} = C\) 时 \(C\) 闭, 乃需验证闭集的公理

        \begin{enumerate}
            \item \(\overline{\varnothing} = \varnothing\), \(\overline{X} \subseteq X\) 从而 \(\overline{X} = X\).
            \item 有限并对 \(A,B\) 闭, \(\overline{A \cup B} = \overline{A} \cup \overline{B} = A \cup B\).
            \item 任意交对 \(A_i\) 闭, 给出 \({(A_i)}_{i \in I}\) 闭, \(\overline{\bigcap_{i \in I} A_i} \cup \overline{A_i} = \overline{\bigcap_{i \in I} A_i \cup A_i} = A_i\),
                    从而有 \(\overline{\bigcap_{i \in I} A_i} \subseteq \bigcap_{i \in I} A_i \subseteq \overline{\bigcap_{i \in I} A_i}\), 于是任意交闭.
        \end{enumerate}

        唯一性源自于给出拓扑结构, 此时 \(\overline{C} = C\) 当且仅当 \(C\) 闭.
    \end{proof}
\end{axiom}

\subsubsection{滤子}

滤子是用来定义极限的.

\begin{definition}[滤子]
    集合 \(X\) 上的一个滤子 (filter) 是一个满足:

    \begin{enumerate}
        \item \(\varnothing \notin \mathcal{F}\)
        \item \(\forall A,B \in \mathcal{F} (A \cap B \in \mathcal{F})\)
        \item \(\forall A \in \mathcal{F} \forall B \subseteq X (A \subseteq B \implies B \in \mathcal{F})\)
    \end{enumerate}

    的 \(\mathcal{F} \subseteq \mathcal{P} (X)\).
\end{definition}

\begin{definition}[滤子基]
    集合 \(X\) 上的一个滤子基 (filter base) 是一个满足:

    \begin{enumerate}
        \item \(\varnothing \notin \mathcal{B}\)
        \item \(\forall A,B \in \mathcal{B} \exists C \in \mathcal{B} (C \subseteq A \cap B)\)
    \end{enumerate}

    的 \(\mathcal{B} \subseteq \mathcal{P} (X)\).

    称 \(\mathcal{B}\) 生成的滤子包含 \(\mathcal{B}\) 的最小的滤子, 也即 \(\mathcal{F} = \{A \subseteq X : \exists B \in \mathcal{B} (B \subseteq A)\}\).
\end{definition}

\begin{definition}[加细]
    假定 \(X\) 上有滤子 \(\mathcal{F},\mathcal{F}^\prime\) 满足 \(\mathcal{F} \subseteq \mathcal{F}^\prime\), 则称 \(\mathcal{F}^\prime\) 是 \(\mathcal{F}\) 的加细. 
\end{definition}

\begin{definition}[超滤]
    不能真加细的滤子称为超滤 (ultrafilter).
\end{definition}

\begin{definition}
    给出滤子 \(\mathcal{F},\mathcal{F}^\prime\), 假设 \(\neg (\exists A \in \mathcal{F} \exists B \in \mathcal{F}^\prime (A \cap B = \varnothing))\),
    则可以定义滤子 \(\mathcal{F} + \mathcal{F}^\prime := \{A \subseteq X : \exists B \in \mathcal{F} \exists C \in \mathcal{F}^\prime (B \cap C \subseteq A)\}\),
    为包含 \(\mathcal{F},\mathcal{F}^\prime\) 的最小滤子.
\end{definition}

\begin{definition}
    对于集合 \(X\) 的子集 \(Y\) 可以定义对应的滤子为 \(\{A \subseteq X : Y \supseteq A\}\).
\end{definition}

\begin{definition}[主滤子]
    对于集合 \(X\) 的元素 \(x\) 可以定义对应的滤子为 \(\{A \subseteq X : x \in A\}\),
    称为 \(x\) 生成的主滤子, 记作 \(\mathcal{F}_x\).
\end{definition}

\begin{corollary}
    一个滤子 \(\mathcal{F}\) 是超滤当且仅当对于任意集合 \(A \subseteq X\), \(X \setminus A \in \mathcal{F} \lor A \in \mathcal{F}\).
\end{corollary}

\begin{corollary}
    主滤子是超滤.
\end{corollary}

\begin{lemma}
    所有滤子都可以加细为超滤.

    \begin{proof}
        利用 \ref{theorem:zorn's lemma}, 线序滤子的并仍是滤子.
    \end{proof}
\end{lemma}

\begin{definition}
    定义一个集合 \(S\) 上的全体滤子的集合为 \(\mathcal{F} (S)\).
\end{definition}

\begin{definition}[收敛空间]
    对集合 \(S\) 定义一个二元关系 \(R \subseteq \mathcal{F} (S) \times S\), 称 \(S\) 为收敛空间 (convergence space), 当满足:

    \begin{enumerate}
        \item 中心的, 即 \((\mathcal{F}_x,x) \in R\).
        \item 同位的, 即对于滤子 \(\mathcal{F} \subseteq \mathcal{F}^\prime\), 如果 \((\mathcal{F},x) \in R\), 则 \((\mathcal{F}^\prime,x) \in R\).
        \item 直接的, 即对于滤子 \(\mathcal{F}, \mathcal{F}^\prime\), 如果 \((\mathcal{F},x) \in R\) 且 \((\mathcal{F}^\prime,x) \in R\), 则 \((\mathcal{F} \cap \mathcal{F}^\prime,x) \in R\).
    \end{enumerate}

    \((\mathcal{F}, x) \in R\) 时称 \(\mathcal{F}\) 收敛到 \(x\), 记作 \(\mathcal{F} \to x\).
\end{definition}

\begin{definition}
    给出映射 \(f : S_1 \to S_2\), 对 \(S_1\) 上的滤子 \(\mathcal{F}\) 定义 \(f (\mathcal{F}) := \{S \subseteq S_2 : \exists A \in \mathcal{F} \land f(A) \subseteq S\}\).
\end{definition}

\begin{corollary}
    如果有 \(f : S_1 \to S_2, g : S_2 \to S_3\), 则有 \(g (f (\mathcal{F})) = (g \circ f) (\mathcal{F})\).
\end{corollary}

\begin{definition}[连续映射]
    对于两个收敛空间 \((S_1,R_1), (S_2,R_2)\), 称 \(f : S_1 \to S_2\) 当且仅当对于任意滤子 \(\mathcal{F} \to x\), 有 \(f (\mathcal{F}) \to f (x)\).
\end{definition}

\begin{definition}[聚点]
    对于集合 \(S\) 上的滤子 \(\mathcal{F}\) 与 \(x \in S\), 如果存在滤子 \(\mathcal{F}^\prime\) 使得 \(\mathcal{F}^\prime \to x\) 且 \(\mathcal{F} \subseteq \mathcal{F}^\prime\), 
    则称 \(x\) 是 \(\mathcal{F}\) 的聚点, 记作 \(\mathcal{F} \leadsto x\).
\end{definition}

\begin{definition}[星性质]
    一个收敛空间若满足对任意滤子 \(\mathcal{F}\), 如果全体含 \(\mathcal{F}\) 的超滤都收敛于 \(x\),
    则 \(\mathcal{F} \to x\), 则称其满足星性质 (star property), 满足星性质的收敛空间称为星收敛空间.
\end{definition}

\begin{corollary}
    对于任何收敛空间, 可以定义星收敛 (star convergence) 为 \(\mathcal{F} \to^{*} x\) 当且仅当 \((\mathcal{F} \subseteq \mathcal{F}^\prime) \implies (\mathcal{F} \leadsto x)\).

    星收敛给出满足星性质的收敛空间.
\end{corollary}

\begin{definition}
    给出拓扑空间 \((X,\mathcal{T})\), 对 \(x \in X\) 定义 \(\mathcal{T}_x\) 为包含 \(x\) 的所有邻域的滤子.
\end{definition}

\begin{definition}
    给出拓扑空间 \((X,\mathcal{T})\), 定义 \(X\) 上的星收敛空间如下, 且定义 \(\mathcal{F} \to x\) 当且仅当 \(\mathcal{T}_x \subseteq \mathcal{F}\).
\end{definition}

\begin{definition}
    给出星收敛空间 \((X,R)\), \(U\) 称 \(x\) 邻域如果对于任意滤子 \(\mathcal{F} \to x\), 均有 \(U \in \mathcal{F}\),
    则称 \(U\) 是 \(x\) 的邻域, 记作 \(U \in \mathcal{T}_x\).
\end{definition}

\begin{definition}
    给出星收敛空间 \((X,R)\), 对集合 \(S \subseteq X\), 定义内部 \(\mathring{S} := \{x \in X : \exists U \in \mathcal{T}_x (U \subseteq S)\}\),
    定义闭包为 \(\overline{S} := \{x \in X : \forall U \in \mathcal{T}_x (U \cap S \neq \varnothing)\}\),
\end{definition}

\begin{definition}
    依赖 \ref{axiom:kuratowski closure}, 星收敛空间给出拓扑空间.

    \begin{proof}
        显然有 \(\overline{\varnothing} = \varnothing\), \(\overline{\overline{A}} = \overline{A}\),
        \(\overline{A \cup B} = \overline{A} \cup \overline{B}\), 利用主滤子可以验证 \(A \subseteq \overline{A}\).
    \end{proof}
\end{definition}

\begin{theorem}
    拓扑空间与星收敛空间是等价的.

    \begin{proof}
        给出拓扑空间 \((X,\mathcal{T})\), 易于验证其闭包和拓扑空间的闭包等价.

        给出星收敛空间 \((X,R)\), 易于验证其对应的拓扑空间定义的邻域与星收敛空间的邻域等价.
    \end{proof}
\end{theorem}

\begin{theorem}
    一个映射在拓扑空间上连续当且仅当其在星收敛空间上连续.

    \begin{proof}
        假使 \(f\) 拓扑空间连续, 则 \(f (x)\) 邻域原像是 \(x\) 邻域, 故 \(f (\mathcal{F}) \to f(x)\).

        假使总是有 \(\mathcal{T}_x \to x\), 故 \(f (\mathcal{T}_x) \to f(x)\), 给出 \(Y\) 开集 \(U\),
        若 \(f (x) \in U\), 则有含 \(f(x)\) 开集 \(f(x) \in V \subseteq U\), 于是 \(f^{-1} (V)\) 含有 \(x\) 的邻域, 
        即原像可以写成这些邻域的并, 从而原像开.
    \end{proof}
\end{theorem}

\begin{definition}
    对于序列 \(\mathbb{N} \to X\), 可以定义其对应的滤子为 \(\{A \subseteq X : \exists n \in \mathbb{N} (x_n \in A)\}\).

    序列 \(\mathbb{N} \to X\) 收敛到 \(x\) 当且仅当其对应的滤子收敛到 \(x\), 记作 \(\lim x_n = x\).
\end{definition}

\subsubsection{度量空间}

\begin{definition}
    给出集合 \(X\), 一个度量 (metric) 是一个映射 \(d : X \times X \to \mathbb{R}\) 满足:

    \begin{enumerate}
        \item 正定性, 即 \(d(x,y) \geq 0\) 且 \(d(x,y) = 0\) 当且仅当 \(x = y\).
        \item 对称性, 即 \(d(x,y) = d(y,x)\).
        \item 三角不等式, 即 \(d(x,y) \leq d(x,z) + d(z,y)\).
    \end{enumerate}
\end{definition}

\begin{definition}
    取幺半范畴 \(\mathbb{R}_{\geq 0}\), 其意味着对象是非负实数, 态射仅在 \(a \geq b\) 时有唯一 \(a \to b\), \(\otimes\) 定义为加法,
    则度量空间可视作 \(\mathbb{R}_{\geq 0}\) - 充实范畴.
\end{definition}

\begin{example}
    取 \(\mathbb{R}\) 上 \(d(x,y) = \abs{x - y}\) 可得到度量空间.
\end{example}

\begin{example}
    取 \(\mathbb{R}^n\) 上 \(d(x,y) = \max \{\abs{x_i - y_i}\}\) 可得到度量空间.
\end{example}

\begin{example}
    度量空间的子空间仍是度量空间
\end{example}

\begin{definition}[开球]
    给出度量空间 \((X,d)\) 中一点 \(x \in X\) 与实数 \(r > 0\), 定义开球 \(B_r (x) := \{y \in X : d(x,y) < r\}\).
\end{definition}

\begin{definition}
    给出度量空间 \((X,d)\), 其上有自然的拓扑 \(\mathcal{T} = \{S \subseteq X : \forall x \in S \exists r > 0 (B_r(x) \subseteq S)\}\).

    \begin{proof}
        首先 \(\varnothing, X \in \mathcal{T}\) 是显然的.

        其次, 给出有限个 \(S_i \in \mathcal{T}\), 任取 \(x \in \bigcap_{i=1}^n S_i\), 则 \(\forall i \exists r_i > 0 (B_{r_i} (x) \subseteq S_i)\),
        取对应的 \(r\) 为 \(r = \mathrm{min}_{i=1}^n r_i\), 则有 \(B_r (x) \subseteq \bigcap_{i=1}^n S_i\).

        最后, 给出任意多个 \(S_i \in \mathcal{T}\), 任取 \(x \in \bigcup_{i \in I} S_i\), 则 \(\exists i \in I (x \in S_i)\),
        于是 \(\exists r > 0 (B_r (x) \subseteq S_i \subseteq \bigcup_{i \in I} S_i)\).
    \end{proof}
\end{definition}

\begin{corollary}
    子度量空间的拓扑与子空间拓扑相同.
\end{corollary}

\begin{lemma}
    度量空间上的拓扑由拓扑基 \(\mathcal{B} = \{B_r (x) : x \in X, r > 0\}\) 生成.

    \begin{proof}
        仅仅是 \ref{lemma:topology generated by base} 的复写.
    \end{proof}
\end{lemma}

\begin{definition}[平常拓扑]
    在一般的分析理论中, 我们在 \(\mathbb{R}^n\) 上取度量空间诱导出的拓扑, 称为平常拓扑 (usual topology), 子集亦取同一度量与子空间拓扑.
\end{definition}

\begin{corollary}
    验证知 \(+ : \mathbb{R}^{2n} \to \mathbb{R}^n\) 连续, \(- : \mathbb{R}^n \to \mathbb{R}^n\) 连续, \(\times : \mathbb{R} \times \mathbb{R}^n \to \mathbb{R}^n\) 连续,
    \(\frac{\bullet}{\bullet} : \mathbb{R} \times \mathbb{R} \setminus \{0\} \to \mathbb{R}\) 连续.
\end{corollary}

\begin{lemma}
    \([0,1]\) 是联通的.

    \begin{proof}
        假设其分拆为开集 \(U,V\), 取 \(x := \sup U\), 则其任意开球均包含 \(U,V\) 中的点, 从而 \(x \notin U,V\) 矛盾.
    \end{proof}
\end{lemma}

\begin{definition}[有界]
    给出 \(Y\) 向度量空间 \((X,d)\) 的映射 \(f : Y \to X\), 如果存在 \(x \in X,r > 0\) 使得 \(f(Y) \subseteq B_r (x)\), 则称 \(f\) 有界 (bounded).
\end{definition}

\begin{definition}
    对于度量空间 \((X,d)\), 全体有界映射 \(Y \to X\) 上可以定义度量 \(d^\prime\) 使得 \(d^\prime (f,g) = \sup_{x \in Y} d(f(x),g(x))\),
    在此度量下, 收敛称为一致收敛 (uniformly convergent).
\end{definition}

\begin{theorem}[Weierstrass 一致收敛定理]
    \setlabel {Weierstrass 一致收敛定理}
    \label {theorem:weierstrass uniform convergence}
    给出拓扑空间 \(X\), 度量空间 \(Y\), 一组连续有界函数 \(f_n : X \to Y\) 一致收敛于 \(f\), 则 \(f\) 仍连续.

    \begin{proof}
        我们证明对于任意 \(\mathcal{F} \to x\), 有 \(f (\mathcal{F}) \to f(x)\).
        任取 \(x\), 对于任意 \(\varepsilon\), 寻求充分大 \(N\) 使得任意 \(n>N\) 有 \(d (f_n,f) \le \varepsilon/3\),
        \(f_n\) 连续意味着存在 \(B_{\varepsilon/3} (f(x))\) 原像开, 注意到 \(\varepsilon/3+\varepsilon/3+\varepsilon/3 \le \varepsilon\), 则对于任意 \(B_{\varepsilon} (f(x))\),
        均有开集 \({f_n}^{-1} (B_{\varepsilon/3} (f(x))) \subseteq f^{-1} (B_{\varepsilon} (f(x)))\), 
        从而 \(f (\mathcal{F}) \to f(x)\).
    \end{proof}
\end{theorem}

\subsection{完备性, 可数性, 分离性}

\subsubsection{完备性}

\begin{definition}[Cauchy 列]
    对一个度量空间 \((X,d)\), 称一列 \(x_n\) 是 Cauchy 列当且仅当对于任意 \(\varepsilon > 0\), 存在 \(N\) 使得 \(n,m \ge N\) 时有 \(d(x_n,x_m) < \varepsilon\).
\end{definition}

\begin{definition}[完备]
    \setlabel {完备}
    \label {definition:complete metric space}
    一个度量空间完备 (complete) 当且仅当其上的 Cauchy 列均收敛.
\end{definition}

\begin{corollary}
    \(X\) 是拓扑空间, \(Y\) 是 \ref{definition:complete metric space} 度量空间, 记 \(C_Y (X)\) 为 \(X\) 到 \(Y\) 上的连续有界函数, 则 \(C_Y (X)\) 是 \ref{definition:complete metric space} 度量空间.

    \begin{proof}
        逐点给出极限, 并利用 \ref{theorem:weierstrass uniform convergence}.
    \end{proof}
\end{corollary}

\begin{lemma}
    \(\mathbb{R}\) 是 \ref{definition:complete metric space} 的.

    \begin{proof}
        构造区间套, 取充分大 \(N\) 使得 \(n,m \ge N\) 时 \(\abs{x_n - x_m} < \varepsilon\),
        从此开始构造区间套, \(a_n = \inf \{x_k : k \ge N\}\), \(b_n = \sup \{x_k : k \ge N\}\),
        存在性有有界和 \ref {theorem:real numbers supremum} 保证, 有 \(x\) 在其交中, 需证明 \(x\) 确为极限.

        任取 \(\varepsilon > 0\), 取充分大 \(N\) 使得 \(n \ge N\) 时有 \(b_n - a_n < \varepsilon\) (Cauchy 列),
        而 \(x_n, x\) 均在 \([a_n,b_n]\) 中, 故有 \(d(x_n,x) < \varepsilon\).
    \end{proof}
\end{lemma}

\begin{definition}
    全体度量空间构成一个范畴 \(\mathbf{Met}\), 全体完备度量空间构成一个范畴, 记为 \(\mathbf{CMet}\), 其中态射是保持距离的映射.
\end{definition}

\begin{definition}[完备化]
    \setlabel {完备化}
    \label {definition:completion of metric space}
    \(\mathbf{CMet}\) 可以嵌入 \(\mathbf{Met}\) 为全忠实子范畴, 且有左伴随函子 \(\mathbf{Met} \to \mathbf{CMet}\), 称为完备化.

    \begin{proof}
        对于度量空间 \((X,d)\), 定义度量空间 \((\overline{X},d^\prime)\), 其元素为 \(X\) 中的全体 Cauchy 列关于等价关系 \(\{x_n\} \sim \{y_n\} \iff \lim d(x_n,y_n) = 0\) 的等价类,
        此关系构成等价关系可由三角不等式说明. 定义 \(d^\prime (\{x_n\},\{y_n\}) = \lim d(x_n,y_n)\), 要证明其是完备度量空间.

        首先由三角不等式可知 \(d^\prime\) 是良定的, 其次 \(d^\prime (\{x_n\},\{y_n\}) = 0\) 定义蕴含 \(\lim d(x_n,y_n) = 0\), 从而 \(\{x_n\} \sim \{y_n\}\),
        如若存在 \(d^\prime (\{x_n\},\{y_n\}) + d^\prime (\{y_n\},\{z_n\}) - d^\prime (\{x_n\},\{z_n\}) = -\varepsilon < 0\), 取充分大 \(N\) 使得 \(d(x_N,y_N) < d^\prime (\{x_n\},\{y_n\}) + \varepsilon/3\), 
        \(d(y_N,z_N) < d^\prime (\{y_n\},\{z_n\}) + \varepsilon/3\) 且 \(d(x_N,z_N) < d^\prime (\{x_n\},\{z_n\}) + \varepsilon/3\), 则有 \(d(x_N,z_N) > d(x_N,y_N) + d(y_N,z_N)\), 矛盾, 最后任意给出 Cauchy 列 \(\{\{x_{m,n}\}_m\}\),
        取对角线元素 \(\{x_{n,n}\}\) 极限即知其完备.

        其伴随性质源于取 \(X\) 向完备度量空间 \(Y\) 的态射 \(f\), 只需取 \(f(\{x_n\}) = \lim f(x_n)\) 即可, 完备化亦给出对保距映射的完备化,
        最后自然性的验证是显见的.
    \end{proof}
\end{definition}

\begin{corollary}
    有自然的嵌入 \(X \to \overline{X}\).
\end{corollary}

\begin{theorem}[压缩映射原理]
    \setlabel {压缩映射原理}
    \label {theorem:contraction mapping principle}
    对于 \ref{definition:complete metric space} 度量空间 \(X\), 任意映射 \(f : X \to X\) 若满足如下条件:
    \[
        \exists \alpha < 1 \forall x,y \in X (d(f(x),f(y)) \le \alpha d(x,y))
    \]
    则称 \(f\) 是压缩映射, 且存在唯一的 \(x \in X\) 使得 \(f(x) = x\).

    \begin{proof}
        任取一点 \(x_0 \in X\), 定义 \(x_n = f^n (x)\), 有 \(d(x_{n+1},x_n) \le \alpha^n d(x_1,x_0)\), 从而 \(x_n\) 是 Cauchy 列, 由完备性知其收敛于某点 \(x\),
        由于 \(f\) 连续, 有 \(f(x) = f(\lim x_n) = \lim f(x_n) = x\), 唯一性源于任取不同于 \(x\) 的点 \(y\), 有 \(d(f(x),f(y)) < d(x,y)\).
    \end{proof}
\end{theorem}

\begin{definition}[极限点]
    对于一个拓扑空间 \(X\) 中的集合 \(A\), \(x \in X\) 是 \(A\) 的极限点 (limit point) 当且仅当任意 \(x\) 的邻域都包含不为 \(x\) 的 \(A\) 中的点.
\end{definition}

\begin{lemma}
    \(\overline{A}\) 是 \(A\) 中所有极限点的集合与 \(A\) 的并.

    \begin{proof}
        对于极限点 \(x\), 任意含 \(x\) 的邻域都包含 \(A\) 中的点, 则任意不含 \(x\) 的开集不含 \(A\), 故 \(x \in \overline{A}\).

        对于 \(x \in \overline{A}, x \notin A\), 任意 \(x\) 的邻域都包含 \(A\) 中的点, 故 \(x\) 是极限点.
    \end{proof}
\end{lemma}

\begin{corollary}
    映射 \(f : X \to Y\) 连续当且仅当 \(f\) 保所有序列极限.

    \begin{proof}
        假设开集原像非开, 则其包含其补的极限点, 矛盾.
    \end{proof}
\end{corollary}

\begin{definition}[稠密]
    \setlabel {稠密}
    \label {definition:dense in topological space}
    对于度量空间 \((X,d)\), \(A \subseteq X\) 稠密 (dense) 当且仅当 \(\overline{A} = X\).
\end{definition}

\begin{example}
    如果 \(\overline{X}\) 是 \(X\) 的 \ref{definition:completion of metric space}, 则 \(X\) 稠密于 \(\overline{X}\).

    \begin{proof}
        所有点都是极限点或 \(X\) 中的点.
    \end{proof}
\end{example}

\begin{definition}[开覆盖]
    对于拓扑空间 \((X,\mathcal{T})\), 其上的集族 \(\mathcal{U} \subseteq \mathcal{T}\) 称为 \(X\) 的开覆盖 (open cover) 当且仅当 \(\bigcup \mathcal{U} = X\),
    其子覆盖 (subcover) 是指 \(\mathcal{V} \subseteq \mathcal{U}\) 且 \(\bigcup \mathcal{V} = X\).
\end{definition}

\subsubsection{可数性}

\begin{definition}[可数性]
    \setlabel {第一可数}
    \label {definition:first countable topological space}
    对于拓扑空间 \((X,\mathcal{T})\), 如果每个点 \(x \in X\) 都有可数邻域基, 则称 \((X,\mathcal{T})\) 是第一可数的 (first countable).

    \setlabel {第二可数}
    \label {definition:second countable topological space}
    如果存在可数拓扑基, 则称 \((X,\mathcal{T})\) 是第二可数的 (second countable).
    
    \setlabel {Lindelöf}
    \label {definition:lindelof topological space}
    如果每个开覆盖都有可数子覆盖, 则称 \((X,\mathcal{T})\) 是 Lindelöf 的.

    \setlabel {可分}
    \label {definition:separable topological space}
    如果存在可数稠密子集, 则称 \((X,\mathcal{T})\) 是可分的 (separable).
\end{definition}

\begin{corollary}
    \ref{definition:second countable topological space} 蕴含 \ref{definition:first countable topological space}.
\end{corollary}

\begin{lemma}
    所有度量空间都是 \ref{definition:first countable topological space} 的.

    \begin{proof}
        有邻域基 \(\{B_{1/n} (x) : n \in \mathbb{Z}_{> 0}\}\).
    \end{proof}
\end{lemma}

\begin{lemma}
    \ref{definition:second countable topological space} 蕴含 \ref{definition:lindelof topological space} 与 \ref{definition:separable topological space}.

    \begin{proof}
        给出可数拓扑基 \(\mathcal{B}\), 对于任意开覆盖 \(\mathcal{U}\), 知每个 \(S \in \mathcal{U}\)
        都可以写成一族 \(B_i \in \mathcal{B}\) 之并, 取可数集 \(\mathfrak{B} := \{B \in \mathcal{B} : \exists S \in \mathcal{U} \land B \subseteq S\}\),
        且每个 \(B\) 均被某个 \(S\) 包含, \(B\) 覆盖 \(X\). 对每个 \(B_i \in \mathfrak{B}\) 寻求对不同 \(i\) 可重复的 \(S_i \in \mathcal{U}\) 使得 \(B_i \subseteq S_i\),
        给出的全体 \(S_i\) 即为可数子覆盖.

        \ref{definition:separable topological space} 只需在 \(\mathcal{B}\) 中的每个 \(B\) 中取一个点即可, 因为不交 \(B\) 的开集均空.
    \end{proof}
\end{lemma}

\begin{lemma}
    度量空间中, \ref{definition:separable topological space}, \ref{definition:lindelof topological space}, \ref{definition:second countable topological space} 等价.

    \begin{proof}
        取 \ref{definition:separable topological space} 度量空间的可数稠密子集 \(S\), 给出 \(\{B_{1/n} (x) : n \in \mathbb{Z}_{> 0}\land x \in S\}\). 需证此集合为可数拓扑基,
        任意给出开球 \(B_r (x)\), 不妨设 \(1/n < r/2\), 则 \(B_{1/n} (x)\) 中有 \(s \in S\), 于是 \(x \in B_{1/n} (s)\),
        对所有开集 \(U\) 中的 \(x\) 均可做此操作, 于是给出了 \(U\) 为上述集合中某些开球的并.

        给出 \ref{definition:lindelof topological space} 度量空间, 给出覆盖 \(\mathcal{U}_n = \{B_{1/n} (x) : x \in X\}\), 有可数子覆盖 \(\mathcal{V}_n = \{B_{1/n} (x) : x \in \mathfrak{V}_n\}\),
        每个 \(\mathfrak{V}_n\) 均可数, 取 \(\mathfrak{V} = \bigcup_{n \in \mathbb{Z}_{> 0}} \mathfrak{V}_n\), 需证 \(\mathfrak{V}\) 为可数稠密子集.
        假若有非空开集交 \(\mathfrak{V}\) 为空, 其中有点 \(x\), 则有邻域 \(B_{1/n} (x) \subseteq U\), 而 \(x \in B_{1/n} (y) \in \mathcal{V}_n\),
        于是 \(y \in U\), 矛盾. 于是 Lindelöf 度量空间 \ref{definition:separable topological space}.
    \end{proof}
\end{lemma}

\begin{lemma}
    \ref{definition:separable topological space} 度量空间的子空间仍然 \ref{definition:separable topological space}.

    \begin{proof}
        \ref{definition:second countable topological space} 性可以直接继承.
    \end{proof}
\end{lemma}

\subsubsection{分离性}

\begin{definition}[分离性]
    \setlabel {\(T_0\)}
    \label {definition:T0 topological space}
    一个空间称 \(T_0\) 或 Kolmogorov 当且仅当对于任意两点 \(x,y\) 存在开集 \(U\) 使得 \(x \in U, y \notin U\) 或 \(y \in U, x \notin U\).

    \setlabel {\(T_1\)}
    \label {definition:T1 topological space}
    一个空间称 \(T_1\) 或 Fréchet 当且仅当对于任意两点 \(x,y\) 存在开集 \(U,V\) 使得 \(x \in U, y \notin U\) 且 \(y \in V, x \notin V\).

    \setlabel {\(T_2\)}
    \label {definition:T2 topological space}
    一个空间称 \(T_2\) 或 Hausdorff 当且仅当对于任意两点 \(x,y\) 存在不交开集 \(U,V\) 使得 \(x \in U, y \in V\).

    \setlabel {\(T_{2\frac{1}{2}}\)}
    \label {definition:T5/2 topological space}
    一个空间称 \(T_{2 \frac{1}{2}}\) 或 Urysohn 当且仅当对于任意两点 \(x,y\) 存在闭集 \(A,B\) 使得 \(x \in A^\circ, y \in B^\circ\) 且 \(A \cap B = \varnothing\).

    \setlabel {\(T_3\)}
    \label {definition:T3 topological space}
    一个空间称 \(T_3\) 或正则 (regular) 当且仅当它 \(T_1\) 且对于任意点 \(x\) 与闭集 \(A\) 使得 \(x \notin A\) 存在开集 \(U,V\) 使得 \(x \in U, A \subseteq V\) 且 \(U \cap V = \varnothing\).

    \setlabel {\(T_{3\frac{1}{2}}\)}
    \label {definition:T7/2 topological space}
    一个空间称 \(T_{3 \frac{1}{2}}\) 或 Tychonoff 当且仅当它 \(T_1\) 且对于任意点 \(x\) 与闭集 \(A\) 使得 \(x \notin A\) 存在连续函数 \(f : X \to [0,1]\) 使得 \(f(x) = 0, f (\{A\}) = \{1\}\).

    \setlabel {\(T_4\)}
    \label {definition:T4 topological space}
    一个空间称 \(T_4\) 或正规 (normal) 当且仅当它 \(T_1\) 且对于任意两个不交闭集 \(A,B\) 存在开集 \(U,V\) 使得 \(A \subseteq U, B \subseteq V\) 且 \(U \cap V = \varnothing\).

    \setlabel {\(T_5\)}
    \label {definition:T5 topological space}
    一个空间称 \(T_5\) 或完全正规 (completely normal) 当且仅当它任意子集 \(T_4\).

    \setlabel {\(T_6\)}
    \label {definition:T6 topological space}
    一个空间称 \(T_6\) 或完美正规 (perfectly normal) 当且仅当它任意两个不交子集 \(A,B\) 存在连续函数 \(f : X \to [0,1]\) 使得 \(f^{-1} (\{0\}) = A, f^{-1} (\{1\}) = B\).
\end{definition}

\begin{corollary}
    \(X\) 是 \ref{definition:T1 topological space} 空间当且仅当每个 \(\{x\}\) 为闭集.

    \begin{proof}
        对于任意 \(y\), 取开集 \(V_y\) 使得 \(x \notin V_y, y \in V_y\), 则 \(X \setminus V_y\) 为闭集, 于是 \(\bigcap_{y \neq x} (X \setminus V_y) = \{x\}\) 为闭集. 

        反之, 取 \(V = X \setminus \{x\}\) 与 \(U = X \setminus \{y\}\) 即可.
    \end{proof}
\end{corollary}

\begin{lemma}
    \(X\) 是 \ref{definition:T2 topological space} 空间当且仅当每个滤子至多收敛到一个点.

    \begin{proof}
        假设 \(X\) 是 \ref{definition:T2 topological space} 空间, 给出滤子 \(\mathcal{F}\) 收敛到 \(x\) 与 \(y\), 取不交开集 \(U,V\) 使得 \(x \in U, y \in V\),
        于是 \(U,V \in \mathcal{F}\), 与 \(U \cap V = \varnothing \notin \mathcal{F}\) 矛盾.

        假使 \(x\), \(y\) 不可被不交开集分离, 构造滤子 \(\mathcal{F} = \{A \cap B : A \in \mathcal{T}_x \land B \in \mathcal{T}_Y\}\),
        \(\mathcal{F}\) 滤子性源于假定给出 \(A \cap B, A^\prime \cap B^\prime \in \mathcal{F}\), 有 \((A \cap B) \cap (A^\prime \cap B^\prime) = (A \cap A^\prime) \cap (B \cap B^\prime) \in \mathcal{F}\),
        空集不在 \(\mathcal{F}\) 源于不\ref{definition:separable topological space}离的假设, 向上封闭性显然, 而上述滤子同时收敛于 \(x\) 与 \(y\).
    \end{proof}
\end{lemma}

\begin{lemma}
    度量空间皆 \ref{definition:T4 topological space}.

    \begin{proof}
        我们可以定义点到集合的距离 \(d(x,A) = \inf \{d(x,a) : a \in A\}\), 
        于是 \(d(x,A) = 0\) 当且仅当 \(x \in \overline{A}\).

        需证明 \(d(x,A)\) 连续, 考虑到其有三角等式, 即 \(d(x,A) \le d(y,A) + d(x,y)\), 
        于是 \(d(x,A)\) 保序列极限, 即连续.

        给出不交闭集 \(C,D\), 定义如下的连续函数 \(f : X \to [-1,1]\),

        \[
            f(x) = \frac{d(x,C) - d(x,D)}{d(x,C) + d(x,D)}
        \]

        取 \(A = f^{-1} ([-1,0))\), \(B = f^{-1} ((0,1])\), 则 \(A,B\) 不交的分离了 \(C,D\).
    \end{proof}
\end{lemma}

\begin{theorem}[Urysohn 引理]
    \setlabel {Urysohn 引理}
    \label {theorem:urysohn's lemma}
    给出 \ref{definition:T4 topological space} 空间 \(X\), 不交闭集 \(C,D\), 存在连续函数 \(f : X \to [0,1]\) 
    使得 \(f(C) = \{0\}\), \(f(D) = \{1\}\).

    \begin{proof}
        我们以 \([0,1]\) 为例, 简述我们的想法再予以证明.

        在 \([0,1]\) 上, \(\{0\},\{1\}\) 可以由 \([0,\frac{1}{3}), (\frac{2}{3},1]\) 分离, 取 \([\frac{1}{3},\frac{2}{3}]\) 的值为 \(\frac{1}{2}\),
        继续对 \([0,\frac{1}{3})\) 操作, \(\{0\}, [\frac{1}{3}, 1]\) 可以由 \([0,\frac{1}{9}), (\frac{2}{9},1]\) 分离,
        取 \([\frac{1}{9},\frac{2}{9}]\) 的值为 \(\frac{1}{4}\), 以此类推, 我们可以定义一个连续函数, 定义为将其写成三进制小数, 假若出现 \(1\), 则去除其后的数位,
        最后将 \(2\) 改为 \(1\) 并以二进制读取.

        我们对拓扑空间 \(X\) 中的不交闭集 \(C,D\) 做此操作, 定义一族闭集 \(A_\alpha\), 开集 \(B_\alpha\) 使得 \(A_0 = C, B_1 = X \setminus D\),
        其中指标取以二的幂为分母的有理数. 归纳的定义 \(X \setminus B_{(2m + 1)/(2^{n+1})}, A_{(2m + 1)/(2^{n+1})}\), 将 \(A_{m/(2^n)}, X \setminus B_{(m+1)/(2^n)}\) 分离,
        即 \(A_{m/(2^n)} \subseteq X \setminus B_{(m+1)/(2^n)}\) 且 \(X \setminus B_{(m+1)/(2^n)} \subseteq X \setminus A_{m/(2^n)}\).
        依赖 \ref{axiom:NBG Axiom of Global Choice} 此构造可以拓展到 \(n \in \mathbb{N}\) 上, 定义函数 \(f(x) := \sup \{\alpha : x \in A_\alpha\}\).

        下面证明 \(f\) 连续, 给出一收敛至 \(x\) 的列 \(x_n\), 任取指标 \(\alpha\) 则 \(x \in A_\alpha\) 给出 \(x\) 邻域,
        于是存在充分大的 \(N\) 使得 \(n \ge N\) 时有 \(x_n \in A_\alpha\), 于是 \(f(x_n) \le \alpha\). 同理, 任取指标 \(\alpha\)
        使得 \(x \notin A_\alpha\), 则 \(X \setminus B_\alpha\) 给出 \(x\) 邻域, 于是存在充分大的 \(N\) 使得 \(n \ge N\) 时有 \(x_n \in X \setminus B_\alpha\),
        任取小于 \(\alpha\) 的指标 \(\beta\) 均有 \(f(x_n) > \beta\). 由于 \(B_{1/2^n} (a)\) 给出 \(\mathbb{R}\) 上 \(a\) 处一邻域基, 故 \(f(x_n)\) 收敛于 \(f(x)\).
    \end{proof}
\end{theorem}

\begin{theorem}[Tietze 延拓]
    \setlabel {Tietze 延拓}
    \label {theorem:tietze's extension}
    给出 \ref{definition:T4 topological space} 空间 \(X\), 闭集 \(E\) 上的连续函数 \(f : A \to [0,1]\) 可以延拓为 \(X\) 上的连续函数.

    \begin{proof}
        我们给出一列函数 \(X \to [0,1]\), 使其极限为所求函数.

        定义 \(f_0 (x) = 0\), 归纳的定义 \(f_n \in C_\mathbb{R} (X)\) 使得在 \(C_\mathbb{R} (E)\) 中 \(d (f_n, f) \le (2/3)^n\), 且 \(\forall x \in E (f_n (x) \le f(x))\).
        给出闭集 \(A = {(f - f_n)}^{-1} ([0,(1/3) \times {(2/3)}^n])\) 与 \(B = {(f - f_n)}^{-1} ([(2/3) \times {(2/3)}^n,{(2/3)}^n])\) 并用 \ref{theorem:urysohn's lemma} 给出在
        \(X\) 上的连续函数 \(g_n\) 使得 \(g_n (A) = \{0\}\), \(g_n (B) = \{1\}\), 取 \(f_{n+1} = f_n + (1/3) \times (2/3)^n g_n\), 显见 \(d (f_{n+1}, f) \le (2/3)^{n+1}\), \(f_{n+1} \le f\).

        考虑到 \(C_\mathbb{R} (X)\) 是完备的, 于是 \(f_n\) 收敛于一个 \(f\) 在 \(E\) 上限制为给出的 \(f\).
    \end{proof}
\end{theorem}

\begin{lemma}
    有 \(T_6 \implies T_5 \implies T_4 \implies T_{3 \frac{1}{2}} \implies T_3 \implies T_{2 \frac{1}{2}} \implies T_2 \implies T_1 \implies T_0\).

    \begin{proof}
        显然 \(T_4\) 当且仅当有 \ref{theorem:urysohn's lemma}.

        \(T_3 \implies T_{2 \frac{1}{2}}\): 给出 \(x,y\), 有 \(x \in U\), \(y \in V\) 且 \(U \cap V = \varnothing\).
        分离 \(x\) 与 \(X \setminus U\), 得到 \(x \in S\), \(X \setminus U \subseteq T\). 其中 \(U,V,S,T\) 开, 故被 \({(X \setminus T)}^\circ,{(X \setminus U)}^\circ\) 分离.
    \end{proof}
\end{lemma}

\begin{definition}
    \(G_\delta\) 集是指可数个开集的交, \(F_\sigma\) 集是指可数个闭集的并.
\end{definition}

\begin{lemma}
    \ref{definition:lindelof topological space}, \ref{definition:T3 topological space} 空间的闭集都是 \(G_\delta\).

    \begin{proof}
        任取一 \(x \notin A\), 有 \(U_x,V_x\) 分离 \(x\) 与 \(A\), 给出开覆盖 \(A \cup \bigcup_{x \in X \setminus A} V_x\)
        依 \ref{definition:lindelof topological space} 有可数子覆盖, 记作 \(A \cup \bigcup_{n \in \mathbb{N}} V_n\), 于是 \(A = \bigcap_{n \in \mathbb{N}} U_n\).
    \end{proof}
\end{lemma}

\begin{lemma}
    度量空间闭集均为 \(G_\delta\).

    \begin{proof}
        取 \(U_n := \bigcup_{x \in A} B_{1/n} (x)\), 有 \(A = \bigcap_{n \in \mathbb{N}} U_n\).
    \end{proof}
\end{lemma}

\begin{lemma}
    \(G_\delta\) 对于连续函数的原像是 \(G_\delta\).
\end{lemma}

\begin{corollary}
    连续函数 \(f : X \to [0,1]\) 则 \(f^{-1} (\{0\})\) 是 \(G_\delta\).
\end{corollary}

\begin{theorem}
    \label {theorem:G-deltas in T4 topological space has continuous function to 0}
    \ref{definition:T4 topological space} 空间有闭 \(G_\delta\) 集 \(C\),
    则有连续函数 \(f : X \to [0,1]\) 使得 \(f^{-1} (\{0\}) = C\).

    \begin{proof}
        假定 \(C = \bigcap_{n \in \mathbb{N}} U_n\), 依 \ref{theorem:urysohn's lemma} 有 \(f_n : X \to [0,1]\) 使得 \(X \setminus U_n \subseteq f_n^{-1} (\{1\})\),
        取 \(f = \frac{1}{2} \sum_{n \in \mathbb{N}} 2^{-n} f_n\), 有 \(f^{-1} (\{0\}) = C\).
    \end{proof}
\end{theorem}

\begin{corollary}
    \ref{definition:T4 topological space} 的两个不交闭 \(G_\delta\) 可以用一个连续函数分离.

    \begin{proof}
        构造 \(f / (f + g)\) 即可.
    \end{proof}
\end{corollary}

\begin{definition}[Hilbert 方块]
    \setlabel {Hilbert 方块}
    \label {definition:hilbert cube}
    Hilbert 方块是空间 \([0,1]^\mathbb{N}\) 与度量 \(d (x_n, y_n) = \max_{n \in \mathbb{N}} 2^{-n} \abs{x_n - y_n}\).
\end{definition}

\begin{theorem}[Urysohn 度量定理]
    所有 \ref{definition:second countable topological space} 且 \ref{definition:T4 topological space} 的拓扑空间都可以视作 \ref{definition:hilbert cube} 的子空间,
    故可以赋予度量.

    \begin{proof}
        给出一可数拓扑基 \(\mathcal{B}\), 对于每个 \(B_n \in \mathcal{B}\) 给出 \(f_{B_n} : X \to [0,1]\) 使得 \({f_{B_n}}^{-1} (\{0\}) = X \setminus B_n\),
        于是给出映射 \(f : X \to [0,1]^\mathbb{N}, f(x) = (f_{B_n} (x))_{n \in \mathbb{N}}\).

        乃需证明此映射单且子空间拓扑诱导出原拓扑, 单性源于此空间 \ref{definition:T1 topological space} 而必然有一拓扑基分离两点,
        \ref{definition:hilbert cube} 有拓扑基为 \(U_n \times {[0,1]}^{\{n+1, n+2, \cdots\}}\), 其中 \(U_n\) 为 \({[0,1]}^n\) 的开集,
        故 \ref{definition:hilbert cube} 拓扑基的原像只需考虑有限个坐标从而开, 而原空间的拓扑基又可由 \({[0,1]}^{n - 1} \times (0,1] \times {[0,1]}^{\{n+1, n+2, \cdots\}}\) 给出,
        故子空间拓扑给出原空间拓扑. 
    \end{proof}
\end{theorem}

\begin{lemma}[Tychonoff]
    \ref{definition:T3 topological space} 且 \ref{definition:lindelof topological space} 的空间 \ref{definition:T4 topological space}.

    \begin{proof}
        对 \(A\) 中点 \(x\) 与 \(B\) 应用 \ref{definition:T3 topological space}, 得到 \(x \in U_x\),
        \(U_x\) 与 \(X \setminus A\) 覆盖 \(X\), 故有可数集 \(U_n\) 覆盖 \(A\), 对称的寻求 \(V_n\) 覆盖 \(B\).
        显见 \(\overline{U_n} \cap B = \varnothing\), \(\overline{V_n} \cap A = \varnothing\).

        构造开集 \(U = \bigcup (U_n \setminus \bigcup_{k \le n} \overline{V_k})\) 与 \(V = \bigcup (V_n \setminus \bigcup_{k \le n} \overline{U_k})\),
        满足 \(U \cap V = \varnothing\), \(A \subseteq U\), \(B \subseteq V\).
    \end{proof}
\end{lemma}

\begin{corollary}[Tychonoff 度量定理]
    \ref{definition:second countable topological space} 与 \ref{definition:T3 topological space} 的空间可被嵌入 \ref{definition:hilbert cube}.
\end{corollary}

\subsection{紧致性}

\subsubsection{紧}

\begin{definition}[紧]
    \setlabel {紧}
    \label {definition:compact topological space}
    一个拓扑空间称紧 (compact) 当且仅当其上任何超滤都有极限.
\end{definition}

\begin{definition}[有限交性]
    \setlabel {有限交性}
    \label {definition:finite intersection property}
    一个拓扑空间有有限交性 (finite intersection property) 当且仅当给出一族闭集 \(C_\alpha\),
    假使对于任意有限的指标集 \(I\) 有 \(\bigcap_{\alpha \in I} C_\alpha \neq \varnothing\), 则有 \(\bigcap_{\alpha \in \mathbb{N}} C_\alpha \neq \varnothing\).
\end{definition}

\begin{corollary}
    一个拓扑空间有 \ref{definition:finite intersection property} 当且仅当其上任何有限开覆盖都有有限子覆盖.
\end{corollary}

\begin{lemma}
    \ref{definition:compact topological space} 当且仅当 \ref{definition:finite intersection property}.

    \begin{proof}
        假使 \(X\) 无 \ref{definition:finite intersection property}, 则给出一族闭集 \(C_\alpha\) 使得有限交空且
        \(\bigcap_{\alpha} C_\alpha = \varnothing\), 于是给出滤子 \(\mathcal{F} = \{S \subseteq X : \exists \abs{I} < \aleph_0 \land S \supseteq \bigcap_{\alpha \in I} C_\alpha\}\),
        使其加细成为超滤, 下证此超滤无极限点. 若某一个滤子既包含 \(x\) 的全体邻域又包含 \(\mathcal{F}\), 由于 \(\bigcap_\alpha C_\alpha = \varnothing\),
        必然有某个 \(\alpha\) 使得 \(x \notin C_\alpha\), 于是 \(X \setminus C_\alpha, C_\alpha\) 均在此滤子中, 违反滤子元素交的性质.

        反之对于有 \ref{definition:finite intersection property} 的空间, 给出无极限的超滤 \(\mathcal{F}\),
        由于 \(x\) 不是极限点, 必然有开邻域 \(U_x \notin \mathcal{F}\), \(\mathcal{F}\) 是超滤, 故不能与滤子 \(\mathcal{N} := \{S \subseteq X : S \supseteq U_x\}\) 合并, 
        于是有 \(A \in \mathcal{F} \land A \cap U_x = \varnothing\), 故 \(X \setminus U_x \in \mathcal{F}\), 于是给出了一个不含 \(x\) 的闭集. 取所有这样的闭集,
        根据 \ref{definition:finite intersection property} 有非空交 \(\bigcap_{k = 1}^n U_{x_k}\), 于是给出了 \(\varnothing = \bigcap_{k = 1}^n U_{x_k} \in \mathcal{F}\),
        矛盾.
    \end{proof}
\end{lemma}

\begin{corollary}
    一个空间 \ref{definition:compact topological space} 且 \ref{definition:T2 topological space} 当且仅当其上任意超滤恰有一个极限点.
\end{corollary}

\begin{lemma}
    \ref{definition:compact topological space} 集的闭子集 \(C\) 仍 \ref{definition:compact topological space}.

    \begin{proof}
        运用 \ref{definition:finite intersection property}, 子空间拓扑闭集在原空间仍闭.
    \end{proof}
\end{lemma}

\begin{lemma}
    \ref{definition:T2 topological space} 且 \ref{definition:compact topological space} 则 \ref{definition:T4 topological space}.

    \begin{proof}
        先证明此空间 \ref{definition:T3 topological space}, 任取 \(x\) 与闭集 \(A\),
        对 \(A\) 中任意点 \(y\) 均有开集 \(x \in U_y\), \(y \in V_y\), \(V_y\) 覆盖 \(A\), 于是有有限子覆盖 \(V_{y_1}, V_{y_2}, \cdots, V_{y_n}\),
        取 \(U = \bigcap_{i = 1}^n U_{y_i}\), \(V = \bigcup_{i = 1}^n V_{y_i}\), 显见 \(U,V\) 分离 \(x\) 与 \(A\).

        同理, 证明此空间 \ref{definition:T4 topological space}, 给出不交闭集 \(A,B\), 对 \(B\) 中任意点 \(y\) 均有开集 \(A \subseteq U_y\), \(y \in V_y\), \(V_y\) 覆盖 \(B\),
        于是有有限子覆盖 \(V_{y_1}, V_{y_2}, \cdots, V_{y_n}\), 取 \(U = \bigcup_{i = 1}^n U_{y_i}\), \(V = \bigcap_{i = 1}^n V_{y_i}\), 显见 \(A \subseteq U\), \(B \subseteq V\).
    \end{proof}
\end{lemma}

\begin{corollary}
    对于 \ref{definition:compact topological space}, \ref{definition:T2 topological space} 空间, 任意两点 \(x,y\) 均有 \(f : X \to [0,1]\) 使得 \(f^{-1} (\{0\}) = \{x\}\), \(f^{-1} (\{1\}) = \{y\}\).

    \begin{proof}
        运用 \ref{theorem:urysohn's lemma}.
    \end{proof}
\end{corollary}

\begin{lemma}
    \label {lemma:T2 compactness implies closedness}
    \ref{definition:T2 topological space} 空间的 \ref{definition:compact topological space} 集均为闭集.

    \begin{proof}
        对于不闭的 \(D\), 给出 \(x \in D \setminus \overline{D}\),
        取滤子为 \(D\) 中的全体 \(x\) 邻域, 其不能加细为有极限超滤因为对于任意点 \(y\) 均有
        \(x,y\) 的不交邻域导致 \(\varnothing\) 在滤子中.
    \end{proof}
\end{lemma}

\begin{definition}[弱序列紧]
    \setlabel {弱序列紧}
    \label {definition:weakly sequentially compact topological space}
    一个拓扑空间称弱序列紧 (weak sequentially compact) 当且仅当其上任何序列都有极限点.
\end{definition}

\begin{definition}[序列紧]
    \setlabel {序列紧}
    \label {definition:sequentially compact topological space}
    一个拓扑空间称序列紧 (sequentially compact) 当且仅当其上任何序列都有收敛子列.
\end{definition}

\begin{corollary}
    在 \ref{definition:first countable topological space} 且 \ref{definition:T2 topological space} 空间, 
    \ref{definition:sequentially compact topological space} 与 \ref{definition:weakly sequentially compact topological space} 等价.
\end{corollary}

\begin{definition}[完全有界]
    \setlabel {完全有界}
    \label {definition:totally bounded topological space}
    一个度量空间称完全有界 (totally bounded) 当且仅当对于任意 \(\varepsilon > 0\) 有有限个 \(B_\varepsilon (x)\) 开球覆盖整个空间.
\end{definition}

\begin{lemma}
    \ref{definition:compact topological space} 且 \ref{definition:first countable topological space} 则 \ref{definition:sequentially compact topological space}.

    \begin{proof}
        给出无收敛子列的序列 \(x_n\), 则对任意一点 \(x \in X\), 取 \(x\) 点的可数基 \(V_k\), 有开集套 \(U_k = \bigcap (V_1 \cap V_2 \cap \cdots \cap V_k)\),
        其中必然有一员 \(U_x\) 使得 \(U_x\) 仅仅包含有限个 \(x_n\).
        取所有 \(U_x\), 其中有有限子覆盖 \(U_{1}, U_{2}, \cdots, U_{m}\), 于是原空间中仅仅包含有限个 \(x_n\), 矛盾.
    \end{proof}
\end{lemma}

\begin{lemma}
    \ref{definition:compact topological space} 则 \ref{definition:weakly sequentially compact topological space}

    \begin{proof}
        同理, 给出无极限点的序列 \(x_n\), 对任意一点 \(x \in X\) 有 \(U_x\) 使得 \(U_x\) 仅仅包含 \(\{x\} \cup \{x_n\}\),
        取所有 \(U_x\), 其中有有限子覆盖 \(U_{1}, U_{2}, \cdots, U_{m}\), 于是原空间中仅仅包含有限个 \(x_n\), 矛盾.
    \end{proof}
\end{lemma}

\begin{lemma}
    \ref{definition:totally bounded topological space} 则 \ref{definition:separable topological space}.

    \begin{proof}
        对 \(\varepsilon = 1/n\) 对每个 \(n \in \mathbb{Z}_{> 0}\) 对有限个开球的中心取并, 得到可数集, 显见其稠密.
    \end{proof}
\end{lemma}

\begin{lemma}
    \ref{definition:complete metric space} 且 \ref{definition:totally bounded topological space} 则 \ref{definition:sequentially compact topological space}.

    \begin{proof}
        定义区域 \(X_0 = X\), 对每个 \(n \in \mathbb{Z}_{> 0}\) 递归的定义区域 \(X_n\) 使其中有无数个序列中的点,
        给出 \(X\) 的有限子覆盖 \(B_{1/n} (x_k)\), 亦覆盖 \(X_{n-1}\), 于是存在一个 \(x_{n_k}\) 使得 \(B_{1/n} (x_{n_k}) \cap X_{n-1}\)
        中有无数个序列中的点, 定义 \(X_n = B_{1/n} (x_{n_k}) \cap X_{n-1}\).

        注意到 \(X_n\) 给出区间套且其中任意两点的距离小于 \(2/n\), 取子序列 \(x_{n_k}\) 使得 \(x_{n_k} \in X_k\) 且 \(n_{k} > n_{k-1}\), 此列是 Cauchy 列,
        依完备性收敛于一点 \(x\).
    \end{proof}
\end{lemma}

\begin{lemma}
    \ref{definition:compact topological space} 度量空间都是 \ref{definition:complete metric space} 且 \ref{definition:totally bounded topological space} 的.

    \begin{proof}
        考察该度量空间的 \ref{definition:completion of metric space}, 利用 \ref{lemma:T2 compactness implies closedness} 证明其闭, 于是 \ref{definition:complete metric space}, 完全有界是显然的.
    \end{proof}
\end{lemma}

\begin{definition}[可数紧]
    \setlabel {可数紧}
    \label {definition:countably compact topological space}
    如果对于任意可数开覆盖都有有限子覆盖, 则称此空间可数紧, 由此可定义对某个基数的紧性.
\end{definition}

\begin{lemma}
    \label {lemma:sequentially compactness implies countably compactness}
    \ref{definition:sequentially compact topological space} 则 \ref{definition:countably compact topological space}.

    \begin{proof}
        我们对不 \ref{definition:compact topological space} 的空间给出一可数无有限子覆盖的开覆盖 \(V_m\),
        归纳的取区域与点 \(U_0 = \varnothing\), 每次取极小的 \(k\) 使得 \(V_k \setminus U_{n-1} \neq \varnothing\), 取 \(x_n \in V_k \setminus U_{n-1}\),
        令 \(U_n = U_{n-1} \cup \{x_n\}\), 因为 \(V_m\) 无有限子覆盖, 于是此构造存在无限个 \(x_n\), 于是给出了无收敛子列的序列, 因为任取点 \(y\), 
        有 \(y \in U_n\) 对充分大的 \(n\), 而 \(U_n\) 有有限个 \(x_n\), 于是必不收敛于 \(y\), 矛盾.
    \end{proof}
\end{lemma}

\begin{corollary}
    \ref{definition:sequentially compact topological space} 且 \ref{definition:lindelof topological space} 则 \ref{definition:compact topological space}.

    \begin{proof}
        运用 \ref{lemma:sequentially compactness implies countably compactness}.
    \end{proof}
\end{corollary}

\begin{example}
    去除 \ref{definition:lindelof topological space} 的条件, 则上述引理不成立,
    考察所有可数基数, 赋予 \(\{0\}\) 与开区间作为拓扑基, 于是此空间不 \ref{definition:compact topological space} 且 \ref{definition:sequentially compact topological space}.

    \begin{proof}
        不紧只需给出 \([0,\alpha)\) 作为开覆盖. 序列紧基于考察对可数列 \(x_n\), 取 \(\bigcup x_n\),
        此为上界, 于是存在序数 \(\alpha \le \bigcup x_n\), 使得 \(\alpha\) 是最小的使得 \([0,\alpha)\) 中有无穷个 \(x_n\),
        的序数, 于是给出了收敛列为全体小于 \(\alpha\) 的序数, 收敛于 \(\alpha\).
    \end{proof}
\end{example}

\begin{lemma}
    \ref{definition:sequentially compact topological space} 的度量空间 \ref{definition:totally bounded topological space} 且 \ref{definition:complete metric space}.

    \begin{proof}
        \ref{definition:complete metric space} 是显然的, \ref{definition:totally bounded topological space} 基于给出任意 \(\varepsilon > 0\) 有有限个 \(B_\varepsilon (x)\) 覆盖整个空间,
        若否, 归纳的给出 \(x_n\) 使得 \(x_n \notin \bigcup_{k = 1}^{n-1} B_{\varepsilon} (x_k)\), 此列无收敛子列, 矛盾.
    \end{proof}
\end{lemma}

\begin{lemma}
    \ref{definition:complete metric space} 的 \ref{definition:totally bounded topological space} 的度量空间是 \ref{definition:compact topological space} 的.

    \begin{proof}
        \ref{definition:totally bounded topological space} 必然是 \ref{definition:lindelof topological space}, 于是 \ref{definition:compact topological space},
        又是 \ref{definition:sequentially compact topological space}, 故 \ref{definition:compact topological space}.
    \end{proof}
\end{lemma}

\begin{corollary}
    对于度量空间, 条件 \ref{definition:compact topological space}, \ref{definition:weakly sequentially compact topological space}, \ref{definition:sequentially compact topological space}, 
    \ref{definition:complete metric space} 且 \ref{definition:totally bounded topological space} 等价.
\end{corollary}

\begin{corollary}
    \(\mathbb{R}\) 上有界闭集是 \ref{definition:compact topological space} 的度量空间.

    \begin{proof}
        显然是 \ref{definition:complete metric space} 且 \ref{definition:totally bounded topological space}.
    \end{proof}
\end{corollary}

\begin{theorem}[Tychonoff]
    \setlabel {Tychonoff 定理}
    \label {theorem:tychonoff's theorem}
    \ref{definition:compact topological space} 空间积仍 \ref{definition:compact topological space}.

    \begin{proof}
        积给出 \(\pi_i : \prod \alpha(i) \to \alpha(i)\), 对 \(\prod \alpha(i)\) 给出超滤 \(\mathcal{F}\),
        显见在 \(\alpha(i)\) 上亦给出了滤子 \(\pi_i \mathcal{F}\), 亦为超滤, 于是有极限 \(x_i\). 下证 \(x = (x_i)\) 为 \(\mathcal{F}\) 极限.

        考察 \(x\) 处的邻域基, 给出 \(\alpha (i_n)\) 中的 \(x\) 邻域 \(U_n\), 考察积拓扑中的开集 \(U = \bigcap \pi_{i_n}^{-1} (U_n)\),
        由于 \(\pi_{i_n} \mathcal{F}\) 有极限点 \(x_{i_n}\), 故必存在 \(V_n \in \mathcal{F}\) 满足 \(\pi_{i_n} (V_n) \subseteq U_n\), 于是 \(V = \bigcap V_n \in \mathcal{F}\)
        且 \(V \subseteq U\), 于是 \(x\) 为 \(\mathcal{F}\) 极限.
    \end{proof}
\end{theorem}

\begin{corollary}
    \ref{definition:compact topological space} \ref{definition:T2 topological space} 空间的极限仍 \ref{definition:compact topological space} \ref{definition:T2 topological space}.

    \begin{proof}
        \ref{definition:T2 topological space} 实属显然, \ref{definition:compact topological space} 上述定理证明过程中的 \(x_i\) 唯一故必然相容.
    \end{proof}
\end{corollary}

\begin{corollary}
    \ref{definition:hilbert cube} 是 \ref{definition:compact topological space} 的拓扑空间.
\end{corollary}

\begin{definition}
    定义 \(\mathbf{Haus}\) 为 \ref{definition:T2 topological space} 的拓扑空间, 态射为连续映射的范畴.

    定义 \(\mathbf{CHaus}\) 为 \ref{definition:compact topological space} \ref{definition:T2 topological space} 的拓扑空间, 态射为连续映射的范畴.
\end{definition}

\subsubsection{局部紧, 紧化}

\begin{definition}[局部]
    \setlabel {局部}
    \label {definition:locally of topological space}
    一个性质称局部的 (locally) 当且仅当其在每一点的都有该点的某个邻域上都有此性质.
\end{definition}

\begin{definition}[局部紧]
    \setlabel {局部紧}
    \label {definition:locally compact topological space}
    一个拓扑空间在 \(x\) 处称局部紧 (locally compact) 当且仅当 \(x\) 处有紧邻域.
    一个拓扑空间称局部紧 (locally compact) 当且仅当其每一点都局部紧.
\end{definition}

\begin{definition}[紧化]
    \setlabel {紧化}
    \label {definition:compactification of topological space}
    \setlabel {单点紧化}
    \label {definition:one-point compactification of topological space}
    给出 \ref{definition:compact topological space} \ref{definition:T2 topological space} 空间 \(Y\), 如果有一真 \ref{definition:dense in topological space} 子空间 \(X \subseteq Y\),
    则称 \(Y\) 是 \(X\) 的紧化 (compactification), 如果 \(Y \setminus X\) 是单元集, 则称 \(Y\) 是 \(X\) 的单点紧化 (one-point compactification).
\end{definition}

\begin{lemma}
    \(X\) 是 \ref{definition:locally compact topological space} \ref{definition:T2 topological space} 空间当且仅当其有 \ref{definition:one-point compactification of topological space}.
    且此 \ref{definition:one-point compactification of topological space} 是唯一的.

    \begin{proof}
        首先我们构造这样一个 \ref{definition:one-point compactification of topological space}, 定义 \(Y = X \cup \{\infty\}\),
        \(\infty\) 为任意与 \(X\) 独立的元素, 需给出 \(\infty\) 处的邻域基, 定义为 \(\mathcal{T}_\infty = \{Y \setminus K : K \text{紧} \land K \subseteq X\}\).

        此确为拓扑基, 因为 \ref{definition:compact topological space} 的有限并仍 \ref{definition:compact topological space}, 需证明 \(Y\) 是 \ref{definition:T2 topological space}, \ref{definition:compact topological space} 空间.
        对于任意一点 \(x \in X\), \(x\) 处有紧邻域 \(K\), 于是 \(Y \setminus K\) 是 \(\infty\) 处的邻域, 故 \ref{definition:T2 topological space}, 而对于任意开覆盖 \(U_\alpha\),
        必然有一项含有 \(\infty\), 而剩余部分是 \ref{definition:compact topological space}, 故有有限子覆盖, 故 \(Y\) 是 \ref{definition:compact topological space} 的.

        另一个方向, 给出 \ref{definition:compact topological space} \ref{definition:T2 topological space} 空间 \(Y\) 与 \(Y \setminus X = \{\infty\}\),
        由于 \ref{definition:T2 topological space} 均可给出一不含 \(\infty\) 闭集, 故 \ref{definition:locally compact topological space}.
    \end{proof}
\end{lemma}

\begin{example}
    对 \(\mathbb{R}^n\) 给出 \(\mathbb{R}^n \cup \{\infty\}\) 为其单点紧化, 其同胚于 \(\mathbb{S}^n := \{(x_i) \in \mathbb{R}^{n+1} : \sum_{i = 1}^{n+1} {x_i}^2 = 1\}\). 
\end{example}

\begin{lemma}
    \ref{definition:locally compact topological space} \ref{definition:T2 topological space} 的开子空间和闭子空间仍 \ref{definition:locally compact topological space}.

    \begin{proof}
        对于闭子空间, 任取 \(x\) 处的 \ref{definition:compact topological space} 邻域 \(K\), 有 \(K \cap A\) 为 \(x\) 处的紧邻域, 故 \(A\) 为局部紧.

        对于开子空间 \(U\), 取一 \(x\) 处 \ref{definition:compact topological space} 邻域 \(K\), 
        考察开覆盖此 \(U\) 与 \(K\) 中 \(y \neq x\) 对应的开集 \(U_y\) 满足 \(y \in U_y\), \(x \in V_y\), \(U_y \cap V_y = \varnothing\),
        有有限子覆盖 \(U_{y_1}, U_{y_2}, \cdots, U_{y_n}, U\), 取 \(V = U \setminus \bigcup_{i = 1}^n U_{y_i}\), 显见 \(V\) 为 \(x\) 处的紧邻域.
    \end{proof}
\end{lemma}

\begin{example}
    考察 \((0,1)\) 可以有 \ref{definition:one-point compactification of topological space} 和 \([0,1]\) 作为 \ref{definition:compactification of topological space},
    考察 \(f : (0,1) \to \mathbb{R}\), 其满足 \ref{definition:compactification of topological space} 后所谓 \(f(1) = f(0)\) 才一定能延拓到 \ref{definition:one-point compactification of topological space} 上.
\end{example}

\begin{theorem}[Stone-Čech]
    \setlabel {Stone-Čech 紧化}
    \label {theorem:stone-cech's compactification}
    对于任意 \ref{definition:T7/2 topological space} \(X\) 有唯一的 \ref{definition:compactification of topological space} \(Y\),
    使得对于任意有界 \(f : X \to \mathbb{R}\) 其可以唯一的延拓到 \(Y\) 上, 且此 \(Y\) 唯一.

    \begin{proof}
        定义 \(S\) 为全体有界函数构成的集合, 给出空间 \([0,1]^S\), 并给出 \(X\) 嵌入 \(\iota\):
        \[
            x \mapsto \{f \mapsto \frac{f(x)}{\sup f - \inf f}\}
        \]
        给出 \(\iota : X \to \overline{\iota (X)}\), 首先这是嵌入且诱导子空间拓扑基于 \ref{definition:T7/2 topological space},
        其中蕴含将两点分离的函数故其单, 其次对于 \([0,1]^S\) 拓扑基中的开集, 择定的是有限个开集之交, 故其确为嵌入.
        其次, 对于任意有界 \(g : X \to \mathbb{R}\) 依定义给出了唯一的延拓 \(g ((x_f)) = x_g\), 唯一性基于其是极限点.

        我们接下来说明 \(Y\) 的唯一性, 假使有另一 \ref{definition:compactification of topological space} 满足上述条件, 按照 \(\iota\) 的定义,
        可以使其嵌入 \([0,1]^S\), 要另 \(Y\) \ref{definition:compact topological space}, 只能是同胚.
    \end{proof}
\end{theorem}

\subsubsection{可度量性, 仿紧}

\begin{definition}[局部有限]
    \setlabel {局部有限}
    \label {definition:locally finite family of topological space}
    一个集合族称局部有限 (locally finite) 当且仅当其每一点都有一个邻域使得其与族中有限个元素相交.
\end{definition}

\begin{definition}[\(\sigma\) - 局部有限]
    \setlabel {\(\sigma\) - 局部有限}
    \label {definition:sigma-locally finite family of topological space}
    一个集合族称 \(\sigma\) - 局部有限 (\(\sigma\) - locally finite) 当且仅当其可以写成可数个局部有限集合族的并.
\end{definition}

\begin{definition}[集族加细]
    \setlabel {加细}
    \label {definition:refinement of sets}
    一个集合族 \(\mathcal{A}\) 称加细 (refinement) 另一个集合族 \(\mathcal{B}\) 当且仅当对于任意 \(A \in \mathcal{A}\) 存在 \(B \in \mathcal{B}\) 使得 \(A \subseteq B\).
\end{definition}

\begin{lemma}
    一个度量空间的任意开覆盖都有 \ref{definition:sigma-locally finite family of topological space} 开覆盖加细.

    \begin{proof}
        给出开覆盖 \(\mathcal{A}\), 依赖 \ref{theorem:well-ordering theorem}, 给出 \(\mathcal{A}\) 上的良序,
        对 \(A \in \mathcal{A}\), 定义 \(S_n (A) := \{x : B_{1/n} (x) \subseteq A\}\), 再定义 \(T_n (A) := S_n (A) \setminus \bigcup_{m < n} S_m (A)\),
        此步给出的 \(T_n (A)\) 两两距离大于 \(1/n\), 再定义 \(E_n (A) := \bigcup_{x \in T_n (A)} B_{1/3n} (x)\), \(E_n(A)\) 开.

        我们证明 \(\bigcup_{n \in \mathbb{Z}_{>0}} E_n (\mathcal{A})\) 给出了 \ref{definition:sigma-locally finite family of topological space} 开覆盖加细,
        首先加细与开显然, \ref{definition:sigma-locally finite family of topological space} 基于 \(A \neq B \implies (E_n (A) \cap E_n (B) = \varnothing)\), 
        覆盖基于对于任意一点 \(x\), 有极小 \(A\) 使得 \(x \in A\), 于是有 \(n\) 使得 \(B_{1/n} (x) \subseteq A\), 于是有 \(x \in T_n (A) \subseteq E_n(A)\).
    \end{proof}
\end{lemma}

\begin{lemma}
    \label {lemma:closure of union of locally finite family of topological space}
    对于 \ref{definition:locally finite family of topological space} 族 \(\mathcal{A}\),
    族 \(\mathcal{B} = \{\overline{A} : A \in \mathcal{A}\}\) 亦为 \ref{definition:locally finite family of topological space},
    有等式 \(\overline{\bigcup \mathcal{A}} = \bigcup \mathcal{B}\).

    \begin{proof}
        任何含 \(x\) 开集只与族中有限个元素相交, 于是同样与闭包中有限个元素相交, 故 \(\mathcal{B}\) 为 \ref{definition:locally finite family of topological space}.

        显见 \(\bigcup \mathcal{B} \subseteq \overline{\bigcup \mathcal{A}}\), 反之, 对于任意点 \(x \in \overline{\bigcup \mathcal{A}}\),
        依局部有限性质知其某邻域与 \(\mathcal{A}\) 中有限个元素相交, 于是其闭包与 \(\mathcal{B}\) 中有限个元素相交, 而闭包保有限并, 故有 \(x \in \bigcup \mathcal{B}\).
    \end{proof}
\end{lemma}

\begin{lemma}
    有 \ref{definition:sigma-locally finite family of topological space} 基的 \ref{definition:T3 topological space} 空间均 \ref{definition:T4 topological space},
    且闭集均为 \(G_\delta\).

    \begin{proof}
        先证明闭集均为 \(G_\delta\), 即证明开集 \(W\) 均为 \(F_\sigma\), 给出 \ref{definition:sigma-locally finite family of topological space} 的拓扑基 \(\mathcal{B} = \bigcup \mathcal{B}_n\),
        对于每一个 \(\mathcal{B}_n\), 构造 \(\mathcal{C}_n := \{X \in \mathcal{B}_n : \overline{X} \subseteq W\}\), 显见 \(\mathcal{C}_n\) 为局部有限, 定义 \(U_n = \bigcup \mathcal{C}_n\),
        由 \ref{definition:T3 topological space} 有等式 \(\bigcup U_n = W\), 而 \(\bigcup U_n \subseteq \bigcup \overline{U_n} \subseteq W\), 于是 \(W\) 为 \(F_\sigma\).
    \end{proof}
\end{lemma}

\begin{theorem}[Nagata-Smirnov 度量化定理]
    \setlabel {Nagata-Smirnov 度量化定理}
    \label {theorem:nagata-smirnov's metrization theorem}
    一个拓扑空间是一个度量空间诱导的当且仅当其 \ref{definition:T3 topological space} 且有 \ref{definition:sigma-locally finite family of topological space} 基.

    \begin{proof}
        对于一个度量空间, 考察开覆盖 \(\{B_{1/n} (x)\}\), 依赖上述引理, 给出了 \ref{definition:sigma-locally finite family of topological space} 开覆盖加细,
        对 \(n \in \mathbb{Z}_{>0}\) 的全体上述加细取并, 给出了 \ref{definition:sigma-locally finite family of topological space} 基.

        仿照 \ref{theorem:urysohn's lemma} 的思路, 我们亦嵌入某个 \(\mathbb{R}^J\) 中, 于是给出了度量化. 给出 \ref{definition:sigma-locally finite family of topological space}
        基 \(\mathcal{B} = \bigcup \mathcal{B}_n\), 对于每一个 \(B \in \mathcal{B}_n\), 给出对应的 \(f_{n,B} : X \to [0,1/n]\) 满足 对任意 \(x \in B\), \(f_{n,B} (x) > 0\),
        而对于任意 \(x \in X \setminus B\), \(f_{n,B} (x) = 0\), 这里是基于 \(X \setminus B\) 是 \(G_\delta\), 可以用 \ref{theorem:G-deltas in T4 topological space has continuous function to 0} 给出连续函数.

        令 \(J = \{(n,B) : n \in \mathbb{Z}_{>0}, B \in \mathcal{B}_n\}\), 和 \ref{theorem:stone-cech's compactification} 类似给出嵌入
        \(\iota : X \to [0,1]^J\), 并且对 \([0,1]^J\) 中的两点给出 \(d ((x_{n,B}), (y_{n,B})) = \sup \{d (x_{n,B}, y_{n,B}) : (n,B) \in J\}\)
        作为度量, 需验证其给出的子空间拓扑同胚于 \(X\).

        给出邻域基中的元素, 注意到 \(\bigcup_{n \leq 1/\varepsilon} \mathcal{B}_n\) \ref{definition:locally finite family of topological space}, 对于任意 \(x \in X\), 存在 \(x\) 开邻域 \(V_x\) 满足其只与
        有限个上述 \(\bigcup_{n \leq 1/\varepsilon} \mathcal{B}_n\) 中的元素相交, 于是给出了 \(V_x \cap \bigcap_{B \cup V_x \neq \varnothing, n \leq 1/\varepsilon} f_{n,B}^{-1} (U_{n,B})\) 为 \(x\) 的邻域,
        于是原像为开集之并故开, 而对于任意基中的元素, 只需考察 \((0,1] \times [0,1]^{J \setminus \{(n,B)\}}\) 开即可.
    \end{proof}
\end{theorem}

\begin{definition}[仿紧]
    \setlabel {仿紧}
    \label {definition:paracompact topological space}
    一个拓扑空间称仿紧 (paracompact) 当且仅当其上任意开覆盖 \(\mathcal{A}\) 都有局部有限开覆盖 \ref{definition:refinement of sets}.
\end{definition}

\begin{corollary}
    \ref{definition:compact topological space} 必然 \ref{definition:paracompact topological space}.
\end{corollary}

\begin{lemma}
    \ref{definition:paracompact topological space} \ref{definition:T2 topological space} 空间 \ref{definition:T4 topological space}.

    \begin{proof}
        先证明其 \ref{definition:T3 topological space}, 给出 \(x\), 闭集 \(A\), 对 \(A\) 中任意点 \(y\) 均有开集 \(x \in U_y\), \(y \in V_y\), \(V_y\) 覆盖 \(A\), 于是有有限子覆盖 \(V_{y_1}, V_{y_2}, \cdots, V_{y_n}\),
        取 \(U = \bigcap_{i = 1}^n U_{y_i}\), \(V = \bigcup_{i = 1}^n V_{y_i}\), 显见 \(U,V\) 分离 \(x\) 与 \(A\).

        同理, 证明其 \ref{definition:T4 topological space}, 给出不交闭集 \(A,B\), 对 \(B\) 中任意点 \(y\) 均有开集 \(A \subseteq U_y\), \(y \in V_y\), \(V_y\) 覆盖 \(B\),
        于是有有限子覆盖 \(V_{y_1}, V_{y_2}, \cdots, V_{y_n}\), 取 \(U = \bigcup_{i = 1}^n U_{y_i}\), \(V = \bigcap_{i = 1}^n V_{y_i}\), 显见 \(A \subseteq U\), \(B \subseteq V\).
    \end{proof}
\end{lemma}

\begin{definition}[单位分解]
    给出加标开覆盖 \(\{U_\alpha\}\) 与对应的加标连续函数 \(\{\phi_\alpha : X \to [0,1]\}\),
    如果满足:

    \begin{enumerate}
        \item 对任意 \(x \in X\) 有 \(\phi_\alpha (x) > 0\) 的只有有限个 \(\alpha\).
        \item 对任意 \(x \in X\) 有 \(\sum \phi_\alpha (x) = 1\).
        \item 对每一个 \(\alpha\), \(\phi_\alpha^{-1} (0,1] \subseteq U_\alpha\).
    \end{enumerate}

    则称其为由 \(\{U_\alpha\}\) 诱导的单位分解 (partition of unity).
\end{definition}

\begin{lemma}[收缩引理]
    \setlabel {收缩引理}
    \label {lemma:shrinking lemma}
    \ref{definition:paracompact topological space} \ref{definition:T2 topological space} 空间的任意加标开覆盖 \(\{U_\alpha\}\) 都可以找到加标 \ref{definition:locally finite family of topological space}
    开覆盖 \(\{V_\alpha\}\), 满足 \(\overline{V_\alpha} \subseteq U_\alpha\).

    \begin{proof}
        考察所有开集 \(B\), 满足 \(\exists \alpha : \overline{B} \subseteq U_\alpha\), 依赖 \ref{definition:T2 topological space} 知全体 \(B\) 给出了开覆盖.
        其有 \ref{definition:locally finite family of topological space} 加细 \(\{B_\beta\}\), 取 \(V_\alpha = \bigcup \{B_\beta : \overline{B_\beta} \subseteq U_\alpha\}\).
        依赖 \ref{lemma:closure of union of locally finite family of topological space} 知 \(\overline{V_\alpha} \subseteq U_\alpha\).
    \end{proof}
\end{lemma}

\begin{theorem}[单位分解定理]
    \setlabel {单位分解定理}
    \label {theorem:partition of unity theorem (paracompact)}
    \ref{definition:paracompact topological space} \ref{definition:T2 topological space} 空间的任意加标开覆盖 \(\{U_\alpha\}\) 都可以找到由其诱导的单位分解.

    \begin{proof}
        运用两次 \ref{lemma:shrinking lemma}, 首先给出加标 \ref{definition:locally finite family of topological space} 开覆盖 \(\{V_\alpha\}\), 满足 \(\overline{V_\alpha} \subseteq U_\alpha\),
        再给出加标 \ref{definition:locally finite family of topological space} 开覆盖 \(\{W_\alpha\}\), 满足 \(\overline{W_\alpha} \subseteq V_\alpha\), 于是利用 \ref{theorem:urysohn's lemma} 找出 \(\psi_\alpha : X \to [0,1]\),
        满足 \(\psi_\alpha (W_\alpha) = \{1\}\), \(\psi_\alpha (X \setminus V_\alpha) = \{0\}\).

        定义 \(\phi_\alpha = \frac{\psi_\alpha}{\sum_\alpha \psi_\alpha}\), 显见其满足单位分解的条件.
    \end{proof}
\end{theorem}

\subsection{函数空间}

\subsubsection{紧开拓扑}

\begin{definition}[紧开拓扑]
    \setlabel {紧开拓扑}
    \label {definition:compact-open topology}
    给出拓扑空间 \(X,Y\) 对紧集 \(K \subseteq X\), 开集 \(U \subseteq Y\),
    定义 \(S(K,U) := \{f \in \mathrm{Hom}_{\mathbf{Top}} (X,Y) : f K \subseteq U\}\),
    定义全体 \(S(K,U)\) 生成的拓扑为 \(\mathrm{Hom}_{\mathbf{Top}} (X,Y)\) 上的紧开拓扑.
\end{definition}

\begin{corollary}
    在 \(\mathrm{Hom}_{\mathbf{Top}} (X,Y), \mathrm{Hom}_{\mathbf{Top}} (Y,Z),\mathrm{Hom}_{\mathbf{Top}} (X,Z)\) 上赋予紧开拓扑, 则复合映射连续.

    \begin{proof}
        注意到态射保紧, 开集原像紧, 故对于紧集 \(K \subseteq X\), \(U \subseteq Z\), 令 \(\mathcal{E}\) 为全体 \(Y\) 中既紧又开的集合,
        则 \(\circ^{-1} (S(K,U)) = \bigcup_{E \in \mathcal{E}} S(E,U) \times S(K,E)\).
    \end{proof}
\end{corollary}

\begin{theorem}[乘积 - 指数伴随]
    对于 \ref{definition:locally compact topological space} \ref{definition:T2 topological space} 空间 \(Y\),
    有函子伴随 \(- \times Y \dashv \mathrm{Hom}_{\mathbf{Top}} (Y, -)\).

    \begin{proof}
        有平凡的自然变换 \(((x,y) \mapsto z) \mapsto (x \mapsto (y \mapsto z))\).

        对连续函数 \(f : X \times Y \to Z, X \to \mathrm{Hom}_{\mathbf{Top}} (Y,Z)\), 连续意味着对于开集 \(U \subseteq Z\), 有 \(\bigcup_\alpha U_{X,\alpha} \times U_{Y,\alpha}\),
        选定一个紧集 \(K \subseteq Y\), 考察 \(x \in f^{-1}(S(K,U))\), 则 \(\forall i \in I_x (x \in U_{X,\alpha_i})\) 且 \(K \subseteq \bigcup_{i \in I} U_{Y,\alpha_i}\), 也即其有有限子覆盖
        故包含一 \(x\) 开邻域 \(\bigcap_{i=1}^n U_{X,\alpha_i}\). 于是对于任意 \(Y\) (可忽略 \ref{definition:locally compact topological space} \ref{definition:T2 topological space}) 此态射均存在.

        若引入 \ref{definition:locally compact topological space} \ref{definition:T2 topological space} 的条件, 对连续函数 \(f : X \to \mathrm{Hom}_{\mathbf{Top}} (Y,Z), X \times Y \to Z\), 
        给出开集 \(U \subseteq Z\), 对每一点 \(f(x,y) \in U\), 有 \(y\) 的邻域 \(U_y\) 使得 \(f(\{x\} \times U_y) \in U\), 于是可找到一紧邻域 \(K \subseteq U_y\) 使得 \(f(\{x\} \times K) \subseteq U\),
        从而有 \(x\) 的邻域 \(U_x\) 使得 \(f(U_x \times K) \subseteq U\), 寻求开集 \(V \subseteq K\), 于是 \(x \in U_x \times V \subseteq f^{-1} (U)\) 为开集.
    \end{proof}
\end{theorem}

\begin{lemma}
    若 \(X\) \ref{definition:compact topological space} 且 \(Y\) 是度量空间, 则
    紧开拓扑给出一致度量对应的拓扑.

    \begin{proof}
        对于紧开拓扑, \(f\) 有邻域为 \(S(K,U)\) 的有限交, 对于每个 \(S(K,U)\) 注意到 \ref{definition:compact topological space}
        像 \ref{definition:compact topological space} 故其闭, 于是有距离 \(\varepsilon = d (f(K),Y \setminus U)\) 的邻域, 取 \(\varepsilon\) 中极小者, 于是包含一致度量的邻域.

        反之, 考察 \(f\) 处以距离 \(\varepsilon\) 诱导的一致度量的邻域,注意到 \(f(X)\) \ref{definition:compact topological space} 故 \ref{definition:totally bounded topological space},
        于是有有限个 \(\varepsilon/2\) 开球覆盖, 于是有有限个 \(S(f^{-1} (\overline{B_{\varepsilon/2}}),B_\varepsilon)\) 交处于其中.
    \end{proof}
\end{lemma}

\subsubsection{Arzelà-Ascoli 定理}

\begin{definition}[预紧]
    \setlabel {预紧}
    \label {definition:precompact topological space}
    一个度量空间称为预紧 (precompact) 当且仅当其 \ref{definition:completion of metric space} \ref{definition:compact topological space}.

    一个空间 \(X\) 对子集 \(A \subseteq X\) 称为预紧, 当且仅当 \(\overline{A}\) \ref{definition:compact topological space}
\end{definition}

\begin{lemma}
    一个度量空间 \ref{definition:precompact topological space} 当且仅当其任意序列有收敛子列.

    \begin{proof}
        注意到 \ref{definition:completion of metric space} 后 \ref{definition:sequentially compact topological space}, 假使任意序列有收敛子列,
        则其完备化 \ref{definition:complete metric space} 后 \ref{definition:sequentially compact topological space}, 于是其 \ref{definition:compact topological space}.
    \end{proof}
\end{lemma}

\begin{remark}
    我们业已在有界函数空间上定义了一致度量, 对于无界函数, 亦可选取截断 \(1\) 定义度量 \(\sup (d(f(x),g(x)),1)\).
\end{remark}

\begin{definition}[等度连续]
    \setlabel {等度连续}
    \label {definition:equicontinuous family of topological space}
    对于度量空间 \(Y\), \(F = \{f : X \to Y\}\), 其称为在 \(x_0\) 处等度连续, 满足 \(\forall \epsilon > 0 \exists U \in \mathcal{T}_{x_0} \forall f (f(\mathcal{T}_x) \subseteq B_\epsilon (f(x_0)))\).

    如果每个点上皆等度连续, 则称其为等度连续.
\end{definition}

\begin{lemma}
    假定上述所论的 \(X,Y\) 皆 \ref{definition:compact topological space}, 则 \ref{definition:equicontinuous family of topological space} \(F\) \ref{definition:totally bounded topological space}.

    \begin{proof}
        任意选取 \(\varepsilon\), 考察 \(\varepsilon/3\) 给出 \(X\) 的一个邻域开覆盖, 其有有限子覆盖 \(\{U_i\}\),
        选取代表元 \(x_i \in U_i\), 给出 \(Y\) 的一个 \(\varepsilon/6\) 为半径的有限开球覆盖 \(B_{\varepsilon/6} (y_i)\),
        对于任意 \(p : \{x_i\} \to \{y_i\}\), 若存在 \(a \in F\) 使得 \(a(x_i) \in B_{\varepsilon/6} (y_i)\), 则选取其中一个计入 \(\mathcal{E}\),
        我们证明有限集 \(\mathcal{E}\) 中元素的 \(\varepsilon\) 半径的开球覆盖 \(F\).

        任意取 \(f \in F\), 给出了 \(p_f : \{x_i\} \to \{y_i\}\), 考察其对应的函数 \(a_f \in F\), 有 \(d (f(x),a_f(x)) < d (f(x),f(x_i)) + d (f(x_i),a_f(x_i)) + d (a_f(x_i),a_f(x)) < \varepsilon\).
    \end{proof}
\end{lemma}

\begin{definition}
    对于两个度量空间间的一族映射 \(F = \{f : X \to Y\}\), 其称为一致等度连续, 若 \(\forall \epsilon > 0 \exists \delta > 0 \forall f \in F \forall x,y \in X (d_X (x,y) < \delta \implies d_Y (f(x),f(y)) < \epsilon)\).
\end{definition}

\begin{definition}
    对于度量空间 \(Y\), \(F = \{f : X \to Y\}\), 其称为逐点有界, 若任意 \(x \in X\), 集合 \(\{f(x) : f \in F\}\) 有界.
\end{definition}

\begin{definition}[点开拓扑]
    在 \(\mathrm{Hom}_{\mathbf{Top}} (X,Y)\) 上, 定义点开拓扑为包含 \(S(\{x\},U)\) 的最小拓扑, 其中 \(x \in X\), \(U \subseteq Y\) 开.
\end{definition}

\begin{lemma}
    点开拓扑诱导逐点收敛.
\end{lemma}

\begin{definition}[紧收敛拓扑]
    \setlabel {紧收敛拓扑}
    \label {definition:compact convergence topology}
    对于度量空间 \(Y\), 在 \(\mathrm{Hom}_{\mathbf{Top}} (X,Y)\) 上, 定义紧收敛拓扑 \(B_C (f,\varepsilon)\) 生成的拓扑,
    其中 \(C \subseteq X\) 紧, \(f \in \mathrm{Hom}_{\mathbf{Top}} (X,Y)\), \(\varepsilon > 0\), \(B_C (f,\varepsilon) = \{g \in \mathrm{Hom}_{\mathbf{Top}} (X,Y) : \forall x \in C (d_Y (f(x),g(x)) < \varepsilon)\}\).
\end{definition}

\begin{lemma}
    \ref{definition:compact-open topology} 给出 \ref{definition:compact convergence topology}.

    \begin{proof}
        对于某个 \(B_C (f,\varepsilon)\), 有 \(f(C)\) \ref{definition:compact topological space},
        给出 \(f(C)\) 的 \(\varepsilon/3\) 开球覆盖, 有有限子覆盖 \(\{B_{\varepsilon/3} (f(x_i))\}\),
        取 \(S(f^{-1} (\overline{B_{\varepsilon/3} (f(x_i))}),B_{2 \varepsilon/3} (f(x_i)))\), 其交在 \(B_C (f,\varepsilon)\) 中.

        如果给出某个 \(S(K,U)\), \(f(K)\) 为 \(U\) 中闭集, 取 \(\varepsilon = d (f(K),Y \setminus U)\), 于是 \(f \in B_C (f,\varepsilon)\).
    \end{proof}
\end{lemma}

\begin{lemma}
    在 \ref{definition:compact convergence topology} 中收敛当且仅当在任意紧集上一致收敛.
\end{lemma}

\begin{definition}[紧生成]
    \setlabel {紧生成}
    \label {definition:compact generated of topological space}
    一个空间称紧生成当且仅当对于任意紧集 \(C\), \(A \cap C\) 在 \(C\) 中均开则 \(A\) 开.
\end{definition}

\begin{lemma}
    \ref{definition:locally compact topological space} 或 \ref{definition:first countable topological space} 则 \ref{definition:compact generated of topological space}.

    \begin{proof}
        假设 \ref{definition:locally compact topological space}, 考察 \(x \in A\) 的 \ref{definition:compact topological space} 邻域 \(x \in U \subseteq C\), \(x \in U \cap A\) 为开.

        假设 \ref{definition:first countable topological space}, 我们证明如果对于任何 \ref{definition:compact topological space} 集 \(C\), \(B \cap C\) 在 \(C\) 中均闭则 \(B\) 闭,
        注意到给出一点 \(x\), 选取收敛于 \(x\) 而不含 \(x\) 的列 \(x_n\), 其 \ref{definition:compact topological space}, 取 \(C\) 为此列即可.
    \end{proof}
\end{lemma}

\begin{lemma}
    \(X\) \ref{definition:compact generated of topological space} 则 \(f : X \to Y\) 连续当且仅当其在任意 \ref{definition:compact topological space} 集上连续.
\end{lemma}

\begin{lemma}
    \ref{definition:compact topological space} 度量空间 \(X\) 的保距映射 \(f : X \to X\) 总是双射.

    \begin{proof}
        考察 \(y \notin f(X)\), 则存在 \(B_{\varepsilon} (y) \cap f(X) = \varnothing\), \ref{definition:sequentially compact topological space} 于是序列 \(\{f^n (y)\}\) 有收敛子列 \(\{f^{n_k} (y)\}\),
        保距性要求 \(d(f^{n}(x),f^{n+k}(x)) = d(x,f^{k}(x)) \geq \varepsilon\), 矛盾.
    \end{proof}
\end{lemma}

\begin{lemma}
    一致收敛则在 \ref{definition:compact convergence topology} 下收敛, 则逐点收敛.
\end{lemma}

\begin{theorem}[Ascoli]
    \(Y\) 是度量空间, 在 \(\mathrm{Hom}_{\mathbf{Top}} (X,Y)\) 上, 给出 \ref{definition:compact convergence topology},
    给出 \ref{definition:equicontinuous family of topological space} 族 \(F\), 若对每个 \(a\), 集合 \(\{f(a) : f \in F\}\) \ref{definition:precompact topological space},
    则 \(F\) \ref{definition:precompact topological space}.

    \begin{proof}
        给 \(Y^X\) 积拓扑, 给出积 \(\prod C_a = \prod \overline{\{f(a):f \in F\}}\), 有 \(F \subseteq \prod C_a\),
        令 \(G\) 为 \(F\) 在此拓扑下的闭包, 则 \(G \subseteq \prod C_a\) 且 \(G\) 闭, 故 \(G\) \ref{definition:compact topological space}.

        \(G\) \ref{definition:equicontinuous family of topological space}, 对于 \(\varepsilon\) 取 \(x_0\) 邻域 \(U\) 使得 \(\forall x \in U \forall f \in F (d(f(x),f(x_0))) < \varepsilon/3\),
        取 \(g \in G\), 则全体 \(h\) 使得 \(d (g(x),h(x)) < \varepsilon/3 \land d (g(x_0),h(x_0)) < \varepsilon/3\) 开, 故 \(d (g(x),g(x_0)) < \varepsilon\).

        我们证明 \(G\) 在 \ref{definition:compact convergence topology} 下给出与积拓扑一样的拓扑, 任意择定紧集 \(C\), 函数 \(g\) 与 \(\epsilon > 0\),
        我们寻求 \(g \in B \cap G \subseteq B_C (g,\varepsilon)\). 依赖 \(C\) \ref{definition:compact topological space}, 故 \(g(C)\) 可以被有限个半径为 \(\epsilon/3\) 的开球覆盖,
        其球心记作 \(x_i\), 在积拓扑中取 \(\prod_i B_{\epsilon/3} (g(x_i)) \times \prod Y\) 此即所求之 \(B\), 又 \ref{definition:compact convergence topology} 比积拓扑细.

        于是 \(G\) 即所求 \ref{definition:compact topological space} 集.
    \end{proof}
\end{theorem}

\begin{theorem}
    假使 \(X\) \ref{definition:locally compact topological space}, \ref{definition:T2 topological space}, 则 \(F\) \ref{definition:precompact topological space}
    诱导出 \(\{f(a) : f \in F\}\) \ref{definition:precompact topological space} 且 \(F\) \ref{definition:equicontinuous family of topological space}.

    \begin{proof}
        显然 \(\mathrm{id}_{\mathrm{Hom}_{\mathbf{Top}} (X,Y)}\) 连续, 依赖 \ref{definition:compact-open topology} 的性质,
        有 \(f \mapsto f(a)\) 亦连续, 只需证明 \(\overline{F}\) \ref{definition:equicontinuous family of topological space}.

        选取一点 \(a\), 考察含 \(a\) 的 \ref{definition:compact topological space} 邻域 \(A\),
        只需考察其在 \(A\) 上的限制 \(\overline{F_A}\) 即可, 基于 \(A\) \ref{definition:compact topological space},
        \(\overline{F_A}\) 在一致度量下亦 \ref{definition:compact topological space} 故 \ref{definition:totally bounded topological space},
        取在一致度量下的 \(\varepsilon/3\) 开球覆盖, 有有限子覆盖 \(\{B_{\varepsilon/3} (f_i)\}\), 给出开集 \({f_i}^{-1} (B_{\varepsilon/3} (a))\),
        其交仍然是 \(a\) 邻域, 记作 \(U\), 任取 \(g \in \overline{F_A}\), 寻求 \(g \in B_{\varepsilon/3} (f_i)\), 于是任意 \(x \in U\) 有 \(d (g(x),g(a)) < d(g(x),f_i(x)) + d(f_i(x),f_i(a)) + d(f_i(a),g(a)) < \varepsilon\).
    \end{proof}
\end{theorem}

\subsubsection{Weierstrass 近似定理, Bernstein 多项式}

\begin{lemma}[Dirichlet-Heine]
    \setlabel {Dirichlet-Heine 引理}
    \label {lemma:Dirichlet-Heine lemma}
    给出 \ref{definition:compact topological space} 度量空间 \(X\) 与度量空间 \(Y\),
    则连续函数 \(f:X \to Y\) 一致连续.

    \begin{proof}
        对每个点 \(x\) 均有 \(\delta_x\) 使得 \(\forall w \in B_{\delta_x} (x) (d_Y (f(w),f(x)) < \varepsilon/2)\),
        寻求 \(B_{\delta_x/2} (x)\) 的有限子覆盖 \(B_{\delta_{x_i}/2} (x_i)\), 令 \(\delta = \min \delta_{x_i}/2\),
        则对于任意 \(d (x,y) < \delta\), 则有 \(d (x_i,x) < \delta_{x_i}/2\), 从而 \(x,y \in B_{\delta_{x_i}} (x_i)\),
        故 \(d (f(x),f(y)) < d(f(x_i),f(x)) + d(f(x_i),f(y)) < \varepsilon\).
    \end{proof}
\end{lemma}

\begin{definition}[Bernstein 多项式]
    \setlabel {Bernstein 多项式}
    \label {definition:Bernstein polynomials}
    对 \([0,1]\)  上连续函数 \(f\), 定义 \(f\) 的 Bernstein 多项式 
    
    \[
        B_n (f) (x) = \sum_{j = 0}^{n} f (\frac{j}{n}) (\binom{n}{j} x^j {(1-x)}^{n-j})
    \]
\end{definition}

\begin{remark}
    对一个像是 \(\mathbb{R}\) 的连续函数, 记 \(\left\| f \right\|_{\infty}\) 为 \(\sup \abs {f}\).
\end{remark}

\begin{theorem}[Bernstein 近似定理]
    \setlabel {Bernstein 近似定理}
    \label {theorem:Bernstein approximation theorem}
    取 \(\varepsilon > 0\), 则 Bernstein 多项式满足不等式:

    \[
        \left\| f - B_n(f) \right\|_\infty \leq \frac{\left\| f \right\|_\infty}{2 n \varepsilon^2} + \sup_{\left|x-y\right| \leq \varepsilon} \abs {f(x) - f(y)}
    \]

    \begin{proof}
        有下列等式和不等式

        \[
            \begin{aligned}
                \sum_{j = 0}^n j \binom{n}{j} x^j {(1-x)}^{n-j} &= \left. x \frac{\dd}{\dd a} \sum_{j = 0}^n \binom{n}{j} {(x+a)}^j {(1-x)}^{n-j} \right|_{a=0} \\
                &= \left. x \frac{\dd}{\dd a} {(1+a)}^n \right|_{a = 0} = nx \\
                \sum_{j = 0}^n j(j-1) \binom{n}{j} x^j {(1-x)}^{n-j} &= \left. x^2 {\frac{\dd}{\dd a}}^2 \sum_{j = 0}^n \binom{n}{j} {(x+a)}^j {(1-x)}^{n-j} \right|_{a=0} \\
                &= \left. x^2 {\frac{\dd}{\dd a}}^2 {(1+a)}^n \right|_{a = 0} = n(n-1)x^2 \\
                \sum_{j = 0}^n {(x - \frac{j}{n})}^2 \binom{n}{j} x^j {(1-x)}^{n-j} 
                &= x^2 - \frac{2nx}{n} + \frac{n(n-1)x^2 + nx}{n^2} = \frac{x(1-x)}{n} \\
                &\leq \frac{1}{4n} \\ 
            \end{aligned}
        \]

        上述不等式给出 
        \[
            \sum_{\abs{x - \frac{j}{n}} > \varepsilon} \binom{n}{j} x^j {(1-x)}^{n-j} \leq \varepsilon^{-2} \sum_{j = 0}^n {(x - \frac{j}{n})}^2 \binom{n}{j} x^j {(1-x)}^{n-j} \leq \frac{1}{4n\varepsilon^2}
        \]

        于是可以对 \(\abs{x - \frac{j}{n}}\) 与 \(\varepsilon\) 关系进行放缩:

        \[
            \begin{aligned}
                \abs{f(x) - B_n(f)(x)} &= \abs {\sum_{j = 0}^{n} (f(x) - f(\frac{j}{n})) \binom{n}{j} x^j {(1-x)}^{n-j}} \\
                &\leq \sum_{j = 0}^{n} \abs {(f(x) - f(\frac{j}{n})) \binom{n}{j} x^j {(1-x)}^{n-j}} \\
                &= (\sum_{\abs{x - \frac{j}{n}} \leq \varepsilon} + \sum_{\abs{x - \frac{j}{n}} > \varepsilon}) \abs {(f(x) - f(\frac{j}{n})) \binom{n}{j} x^j {(1-x)}^{n-j}} \\
                &\leq \sup_{\abs{x-y} \leq \varepsilon} \abs{f(x) - f(y)} + 2 \left\|f\right\|_\infty \frac{1}{4n\varepsilon^2} \\
            \end{aligned}
        \]

        于是原不等式成立

        \[
            \left\| f - B_n(f) \right\|_\infty \leq \frac{\left\| f \right\|_\infty}{2 n \varepsilon^2} + \sup_{\left|x-y\right| \leq \varepsilon} \abs {f(x) - f(y)}
        \]
    \end{proof}
\end{theorem}

\begin{theorem}[Weierstrass 第一定理]
    \label{theorem:weierstrass' first theorem}
    全体多项式在 \(C([a,b])\) 上稠密, 也即对于任意连续 \(f : [a,b] \to \mathbb{R}\), 存在多项式序列 \(\{p_n\}\) 一致收敛于 \(f\).

    \begin{proof}
        \([a,b]\) 可线性变换至 \([0,1]\), 且保持其上多项式.

        取 \ref{definition:Bernstein polynomials}, 依 \ref{theorem:Bernstein approximation theorem} 两边取极限给出仅含 \(\varepsilon\) 的不等式 \({\lim \sup}_{n \to \infty} \left\| f - B_n(f) \right\| \leq \sup_{\left| x-y \right| \leq \varepsilon} \left|f(x) - f(y)\right|\),
        依赖 \ref{lemma:Dirichlet-Heine lemma} 知 \(f\) 一致连续, 也即 \(B_n(f) \to f\).
    \end{proof}
\end{theorem}

\begin{theorem}[Weierstrass 第二定理]
    \label {theorem:weierstrass' second theorem another}
    全体 \(\exp {\pm i \theta}\) 构成的多项式在赋予一致度量下的 \(\mathrm{Hom}_{\mathbf{Top}} ([0,2\pi]/\{0,2\pi\},\mathbb{C})\) 上稠密, 其中 \(\mathbb{C}\) 上的度量取为 \(\abs {x - y}\).

    \begin{proof}
        见分析基础, Hilbert 空间.
    \end{proof}
\end{theorem}

\begin{theorem}[J.Pál]
    对于 \([0,1]\) 上的连续函数, 上述结果可以优化, \(f\) 可以用整系数多项式逼近当且仅当 \(f(0),f(1) \in \mathbb{Z}\).
    \begin{proof}
        仍然仿照上述证明, 系数改为为取整, 需估算余项大小

        \[
            B_n^\prime (f) (x) = \sum_{j = 0}^{n} \left[{f (\frac{j}{n}) \binom{n}{j}}\right] x^j {(1-x)}^{n-j}
        \]

        \[
            (\delta B_n) (f) (x) = \sum_{j = 0}^{n} \left\{{f (\frac{j}{n}) \binom{n}{j}}\right\} x^j {(1-x)}^{n-j}
        \]

        记其系数如下, 易见不等式:

        \[
            \delta b (f)_{n,j} := {f (\frac{j}{n}) \binom{n}{j}}
        \]

        \[
            \delta b(f)_{n,j} < 1 \leq \begin{cases}
                \frac{1}{n} \binom{n}{j} & j=1,n-1 \\
                \frac{4}{n^2} \binom{n}{j} & 0 < j < n-2
            \end{cases}
        \]

        假使 \(f(0),f(1) \in \mathbb{Z}\), 易知 \(\delta b (f)_{n,0} = \delta b (f)_{n,n} = 0\),
        于是 \(\delta B_n (f) < \frac{6}{n}\) 故一致收敛.
    \end{proof}
\end{theorem}

\begin{theorem}[Weierstrass 近似定理]
    \setlabel {Weierstrass 近似定理}
    \label {theorem:Weierstrass's approximation theorem}
    \(\mathbb{R}^{\nu}\) 中紧集上的 \(\mathbb{R}\) 值函数可以用多项式逼近.

    \begin{proof}
        依旧可以压缩平移到 \({[0,1]}^\nu\) 紧子集, 然后依赖 \ref{theorem:tietze's extension} 给出 \({[0,1]}^\nu\) 上连续函数, 仿照 \ref{definition:Bernstein polynomials} 定义:

        \[
            (B_{n,\nu} f) (x) = \sum_{j_1,\cdots,j_\nu = 0}^{n} f(\frac{j_1}{n},\frac{j_2}{n},\cdots,\frac{j_\nu}{n}) \prod_{k=1}^{\nu} \binom{n}{j_k} {x_k}^{j_k} {(1-x_k)}^{n - j_k}
        \]

        寻相似的方法, 赋予 \(\abs{x} = \sqrt{\sum_{i=1}^{\nu} {x_i}^2}\) 度量, 仍然有:

        \[
            \left\| B_{n,\nu} (f) - f \right\|_\infty \leq \frac{\nu \left\|f\right\|_\infty}{2n\varepsilon^2} + \sup_{\abs{x-y} \leq \varepsilon} \abs{f(x) - f(y)}
        \]
    \end{proof}
\end{theorem}

\begin{definition}[Korovkin 集]
    对于 \ref{definition:compact topological space} \ref{definition:T2 topological space} 空间 \(X\), 称其上有限个 \(\mathbb{R}\) 值连续函数构成的集合 \(\{f_1,f_2,\cdots,f_m\}\) 是 Korovkin 集,
    若存在 \(\mathbb{R}\) 值连续函数 \(a_i (t)\), 使得 \(P(x,t) = \sum_{j=1}^{m} a_j(t) f_j(x)\), 满足 \((P(x,t) \geq 0) \land (P(x,t) = 0 \iff x = t)\).
\end{definition}

\begin{example}
    \(\{1,x,x^2\}\) 是 \([0,1]\) 上的 Korovkin 集, 只需注意到 \((x-t)^2 = 0\) 即可.
\end{example}

\begin{definition}[保正]
    \setlabel {保正}
    \label {definition:positivity-preserving operator}
    一个线性算子 \(B\) 称为保正的, 若对于任意一个处处非负非负的 \(f\), \(B(f)\) 依然处处非负.
\end{definition}

\begin{theorem}[Korovkin]
    \setlabel {Korovkin 定理}
    \label {theorem:Korovkin's theorem}
    对于一个有 Korovkin 集 \(\{f_i\}\) 的 \ref{definition:compact topological space} \ref{definition:T2 topological space} 空间 \(X\), 
    给出 \(\mathrm{Hom}_{\mathbf{Top}} (X,\mathbb{R})\) 一列保正的线性算子 \(\{T_n\}_{n \in \mathbb{N}}\), 如果对每个 \(\{f_i\}\),
    均有 \(\lim_n \left\| T_n f_i - f_i \right\|_\infty = 0\), 则对于任意 \(f\), 均有 \(\lim_n \left\| T_n f - f \right\|_\infty = 0\)
\end{theorem}

\subsubsection{Stone-Weierstrass 定理}

\begin{definition}[分离点]
    一个 \(S \subseteq \mathrm{Hom}_{\mathbf{Top}} (X,\mathbb{R})\), 其分离 \(X\) 上的点,
    使得 \(\forall x,y \in X (x \neq y \implies (\exists f \in S (f(x) \neq f(y))))\), 
    其称为强分离如果 \(\forall x,y \in X (x \neq y \implies (\mathbb{R}^2 = \{(f(x),f(y)) : f \in S\}))\).
\end{definition}

\begin{remark}
    我们记 \(f,g\) 的上界和下界为 \((f \vee g) (x) = \max (f(x),g(x)),(f \wedge g) (x) = \min (f(x),g(x))\).
\end{remark}

\begin{definition}
    一个格是容许上下界的 \(\mathrm{Hom}_{\mathbf{Top}} (X,\mathbb{R})\) 子空间.

    一个线性格是容许上下界的 \(\mathrm{Hom}_{\mathbf{Top}} (X,\mathbb{R})\) 线性子空间.
\end{definition}

\begin{theorem}[Kakutani-Krein]
    给出 \ref{definition:compact topological space} \ref{definition:T2 topological space} 空间 \(X\) 以及 \(L \subseteq \mathrm{Hom}_{\mathbf{Top}} (X,\mathbb{R})\),
    满足 \(L\) 是一个向量格且 \(L\) 强分离 \(X\) 上的点, 则一致度量下 \(L\) 的闭包是 \(\mathrm{Hom}_{\mathbf{Top}} (X,\mathbb{R})\).

    \begin{proof}
        任给 \(f \in \mathrm{Hom}_{\mathbf{Top}} (X,\mathbb{R})\), 对每队 \(x,y \in X\), 均可构造 \(h_{x,y} \in L\) 使得
        \(h_{x,y} (x) = f(x),h_{x,y} (y) = f(y)\), 定义开集 \(U_{x,y} := \{z : f(z) - h_{x,y} (z) < \varepsilon\}\), 固定 \(x\), \(\{U_{x,y} : y \in X \setminus \{x\}\}\) 有有限子覆盖
        \(U_{x,y_i}\), 在此覆盖上取最大值, 得到 \(g_x (z) := \max_i h_{x,y_i} (z)\), 于是 \(\forall z \in X (g_x(z) > f(z) - \varepsilon)\),
        对称的, 对每个 \(x\) 定义开集 \(V_x := \{z : f(z) - g_x (z) > -\varepsilon\}\), 其有有限子覆盖 \(V_{x_i}\), 取 \(l(z) := \min_i g_{x_i} (z)\),
        满足 \(\forall x \in X (f(x) - \varepsilon < l(x) < f(x) + \varepsilon)\).
    \end{proof}
\end{theorem}

\begin{theorem}[Stone-Weierstrass]
    给出 \ref{definition:compact topological space}, \ref{definition:T2 topological space} \(X\), 令 \(S \subseteq \mathrm{Hom}_{\mathbf{Top}} (X,\mathbb{R})\),
    如果 \(S\) 在加法乘法数乘下封闭, 且 \(S\) 分离点, 且 \(1 \in S\), 则一致度量下 \(S\) 的闭包是 \(\mathrm{Hom}_{\mathbf{Top}} (X,\mathbb{R})\).

    \begin{proof}
        蕴涵关于一个函数 \(f\) 的多项式, 于是可以一致逼近 \(\abs{f}\), 从而 \(\overline{S}\) 是向量格,
        依赖 \(1 \in S\) 与分离 \(X\) 中的点, 知道 \(\overline{S}\) 强分离 \(X\) 中的点,
        故 \(\overline{S} = \overline{\overline{S}} = \mathrm{Hom}_{\mathbf{Top}} (X,\mathbb{R})\).
    \end{proof}
\end{theorem}

\subsection{Baire 纲, 拓扑维数}

\subsubsection{Baire 纲}

\begin{definition}[Baire 空间]
    \setlabel {Baire}
    \label {definition:Baire space}
    一个拓扑空间称 Baire 空间 (Baire space) 当且仅当可数个有空内部的闭集的并仍有空内部.
\end{definition}

\begin{corollary}
    一个拓扑空间 \(X\) 是 \ref{definition:Baire space} 空间当且仅当可数个稠密的开集的交的闭包仍稠密.
\end{corollary}

\begin{theorem}[Baire 纲]
    \ref{definition:compact topological space} \ref{definition:T2 topological space} 空间与
    \ref{definition:complete metric space} 度量空间均为 \ref{definition:Baire space} 空间.

    \begin{proof}
        当空间 \ref{definition:compact topological space} \ref{definition:T2 topological space} 时, 任取开集 \(U_0\), 给出可数个闭 \(F_n\) 有空内部, 归纳定义非空开集 \(U_{n+1} = U_n \setminus F_n\),
        得到区间套 \(\{F_n\}\), 定义滤子 \(\mathcal{F} = \{F \subseteq X : \exists n (F_n \subseteq F)\}\),
        依赖 \ref{definition:compact topological space} \ref{definition:T2 topological space}, 其被加细为超滤后有极限 \(x\),
        依赖 \ref{definition:T2 topological space} (\ref{definition:T4 topological space}) 知 \(x \in \bigcap F_n\), 依赖 \(U_0\) 任意性知 \(\bigcap F_n\) 有空内部.

        当空间 \ref{definition:complete metric space} 且有度量时, 非空开集均包含某个开球, 寻上述思路构造开球套 \(\{U_n = B_{r_n} (x_n)\}\), 满足 \(r_n < 1/n\),
        考察数列 \(\{x_n\}\), 依赖完备性知其有极限 \(x\), 而 \(x \in \bigcap U_n\), 依赖 \(U_0\) 任意性知 \(\bigcap F_n\) 有空内部.
    \end{proof}
\end{theorem}

\begin{theorem}
    \ref{definition:locally compact topological space} \ref{definition:T2 topological space} 空间及其 \(G_\delta\) 均为 \ref{definition:Baire space} 空间.

    \begin{proof}
        仿照上述证明, 此处略.
    \end{proof}
\end{theorem}

\begin{lemma}
    \ref{definition:Baire space} 空间的开子空间仍为 \ref{definition:Baire space} 空间.

    \begin{proof}
        需证明一个在子空间 \(Y\) 中有空内部的闭集 \(F\), 其闭包 \(\overline{F}\) 在全空间 \(X\) 中有空内部,
        只需注意到, 若某一点有开集 \(Y \subseteq \overline{F}\), 则有 \(Y \cap U \subseteq F\).
    \end{proof}
\end{lemma}

\begin{example}
    对于连续函数 \(f : \mathbb{R}_{>0} \to \mathbb{R}\), 若对于任意 \(\delta \in \mathbb{R}_{> 0}\), 有 
    \(\lim_{m} f (m \delta) = 0\), 则 \(\lim_{x \to +\infty} f (x) = 0\).

    其中 \(\lim_{x \to +\infty} f (x) = 0\) 表示滤子 \(\mathcal{F} = \{A : \exists x (f((x,+\infty)) \subseteq A)\}\) 有极限 \(0\).

    \begin{proof}
        若否, 则存在 \(\varepsilon > 0\) 使得对任意 \(N \in \mathbb{R}_{>0}\), 存在 \(x > N\) 使得 \(\abs{f(x)} > \varepsilon\),
        考察 \([1,2]\) 作为 \ref{definition:Baire space} 空间, 包含开集族 \(U_N := \{r \in [1,2] : \exists m > N (f(m r) > \varepsilon)\}\),
        稠密源于对任意开集 \((a,b)\), 有 \(N \ge \frac{1}{a - b}\), 于是 \((N a, \infty) \subseteq \bigcup_{n \in \mathbb{Z}_{> 0}} (n a, n b)\),
        其并仍稠密故非空, 于是存在 \(x \in \bigcap_{N \in \mathbb{N}} U_N\), 使得 \(f(m x) > \varepsilon\), 矛盾.
    \end{proof}
\end{example}

\subsubsection{拓扑维度}

\begin{definition}[阶数]
    一个 \(X\) 上的集合有阶数 \(m\) 若其每个点最多被 \(m\) 个集合包含.
\end{definition}

\begin{definition}[拓扑维度]
    \setlabel {拓扑维度}
    \label {definition:dimension of topological space}
    一个拓扑空间 \(X\) 的拓扑维度 (topological dimension) 为 \(\underline{\dim} X\) 当且仅当任意 \(X\) 开覆盖 \(\mathcal{A}\) 都有 \(\underline{\dim} X+1\) 阶数的加细 \(\mathcal{B}\) 
    使得 \(\mathcal{B}\) 也为开覆盖, 且 \(\underline{\dim} X\) 极小.

    若一个空间存在拓扑维数, 则称其为有限维空间 (finite-dimensional space).
\end{definition}

\begin{lemma}
    给出 \(\underline{\dim} X = n\), 则任意 \(X\) 的闭子空间 \(\underline{\dim} Y \leq n\).

    \begin{proof}
        考察含 \(X \setminus Y\) 的 \(X\) 开覆盖即可.
    \end{proof}
\end{lemma}

\begin{lemma}
    如果闭集 \(Y,Z\) 满足 \(X = Y \cup Z\), 则有 \(\underline{\dim} X = \max \{\underline{\dim} Y, \underline{\dim} Z\}\).

    \begin{proof}
        对于开覆盖 \(\mathcal{A}\), 给出加细 \(\mathcal{B}\) 作为 \(X\) 的开覆盖, 使得其在 \(Y\) 上有 \(\underline{\dim} Y+1\) 阶,
        再给出 \(\mathcal{B}\) 的加细 \(\mathcal{C}\) 作为 \(X\) 的开覆盖, 使得其在 \(Z\) 上有 \(\underline{\dim} Z+1\) 阶, 有典范的 \(f : \mathcal{C} \to \mathcal{B}\),
        使得 \(C \subseteq f(C)\), 将 \(f(C)\) 相同的元素合并, 得到 \(\mathcal{D}\) 作为 \(X\) 的开覆盖, \(\mathcal{D}\) 即为 \(X\) 上阶数为 \(\max \{\underline{\dim} Y, \underline{\dim} Z\}\) 的加细.
    \end{proof}
\end{lemma}

以下内容证明略去, 留于流形的理论之中.

\begin{example}
    \(\mathbb{R}^n\) 上的方格覆盖, 考察将 \(\mathbb{R}^n\) 分拆为不带边界的 \({(0,1)}^n\),
    不带边界但是有一定厚度的 \({(0,1)}^{n-1} \times (-\varepsilon,\varepsilon)\) 等, 
    其开覆盖加细总是 \(n+1\) 阶的.
\end{example}

\begin{lemma}
    \(\mathbb{R}^n\) 的紧子空间维数小于等于 \(n\).
\end{lemma}

\begin{theorem}[嵌入定理]
    任何维度有限 \ref{definition:compact topological space} 可度量化的空间 \(X\) 都可以嵌入到 \(\mathbb{R}^{2 \underline{\dim} X + 1}\) 中.
\end{theorem}

\subsection{代数几何扩充 (Unfinished)}

\subsubsection{不可约}

\begin{definition}[分离]
    \setlabel {分离}
    \label {definition:separation of continuous function}
    一个连续映射 \(f\) 称分离 (separated) 当且仅当对于任何不同的 \(x,x^\prime\), \(f (x) = f (x^\prime)\) 
    则存在开集 \(U,V\) 使得 \(x \in U\), \(x^\prime \in V\), \(U \cap V = \varnothing\).
\end{definition}

\begin{definition}[闭映射]
    \setlabel {闭}
    \label {definition:closed map}
    一个映射 \(f : X \to Y\) 称闭 (closed) 当且仅当对于任何闭集 \(A \subseteq X\), \(f (A)\) 闭.
\end{definition}

\begin{lemma}
    \(f : X \to Y\) \ref{definition:separation of continuous function} 当且仅当 \(\Delta := \mathrm{id}_X \times_Y \mathrm{id}_X : X \to X \times_Y X\) 为 \ref{definition:closed map} 映射.
    也即 \(\Delta (X)\) 闭.

    \begin{proof}
        假定 \ref{definition:separation of continuous function}, 任意给出 \((x,x^\prime) \notin \Delta (X)\), 有 \(f (x) = f (x^\prime)\), 
        于是有开集 \(U,V\) 使得 \(x \in U\), \(x^\prime \in V\), \(U \cap V = \varnothing\), 于是 \((x,x^\prime) \in U \times_Y V\), 于是 \(\Delta (X)\) 闭.

        假定 \(\Delta (X)\) 闭, 任意给出 \(f (x) = f (x^\prime)\), 有 \((x,x^\prime) \in \Delta (X)\), 考察 \((x,x^\prime)\)
        的邻域基 \(\{U \times_Y V\}\), 有 \((x,x^\prime) \in U \times_Y V\), 于是有 \(x \in U\), \(x^\prime \in V\), \(U \cap V = \varnothing\).

        \ref{definition:closed map} 等价于 \(\Delta (X)\) 闭是显然的, 至于将闭集 \(C\) 对应到 \(\Delta (C) = C \times_Y C \cap \Delta (X)\) 闭.
    \end{proof}
\end{lemma}

\begin{corollary}
    \ref{definition:T2 topological space} 等价于 \(\Delta (X) := \mathrm{id}_X \times_{\{\bullet\}} \mathrm{id}_X : X \to X \times X\) 为 \ref{definition:closed map} 映射.
\end{corollary}

\begin{definition}[不可约]
    \setlabel {不可约}
    \label {definition:irreducible topological space}
    一个非空拓扑空间称不可约 (irreducible) 当且仅当其不是两个真闭子空间的并.
\end{definition}

\begin{lemma}
    拓扑空间 \(X\) 子空间 \(Y\) \ref{definition:irreducible topological space} 当且仅当 \(\overline{Y}\) \ref{definition:irreducible topological space}.
\end{lemma}

\begin{lemma}
    \(\overline{\{x\}}\) 总是 \ref{definition:irreducible topological space}.
\end{lemma}

\begin{definition}
    \setlabel {一般点}
    \label {definition:generic point}
    如果有 \(\overline{\{y\}} \subseteq \overline{\{x\}}\) 则称 \(x\) 是 \(y\) 的一般化,
    \(y\) 是 \(x\) 的特殊化, 如果 \(\overline{\{x\}} = X\), 则称 \(x\) 为 \ref{definition:generic point}.
\end{definition}

\begin{corollary}
    \ref{definition:T0 topological space} 空间至多有一个 \ref{definition:generic point}.
\end{corollary}

\begin{lemma}
    \ref{definition:irreducible topological space} 空间的开子集仍 \ref{definition:irreducible topological space},
    且保 \ref{definition:generic point}.

    \begin{proof}
        假设有 \(U \subseteq C_1 \cup C_2\), 则 \(((X \setminus U) \cup C_1) \cup ((X \setminus U) \cup C_2) = X\), 
        与 \(X\) \ref{definition:irreducible topological space} 矛盾, 保 \ref{definition:generic point} 只需注意到一般点在任何非空开集中.
    \end{proof}
\end{lemma}

\begin{lemma}
    连续满射 \(f\) 保 \ref{definition:irreducible topological space} 并且保 \ref{definition:generic point}.
\end{lemma}

\begin{lemma}
    \ref{definition:irreducible topological space} 子空间的线序子集有上界, 从而有极大 \ref{definition:irreducible topological space} 子空间.

    \begin{proof}
        取并集, 任意真闭集之并可在某个线序子集某个元中进行分拆得到 \(\varnothing,A\), 由于 \(A\) 会传递为线序集更大元素中某个分拆的子集,
        所以 \(\varnothing\) 必然传递到线序集更大的每个元素, 于是在整体闭集中也为 \(\varnothing\).
    \end{proof}
\end{lemma}

\begin{definition}
    \setlabel {不可约分支}
    \label {definition:irreducible component}
    一个 \ref{definition:irreducible topological space} 的极大子空间称为其不可约分支 (irreducible component), 其必然闭.
\end{definition}

\begin{definition}[升链条件]
    \setlabel {升链条件}
    \label {definition:ascending chain condition}
    一个集合族称满足升链条件 (ascending chain condition) 当且仅当其任意升链都止于某个元素.
    \setlabel {降链条件}
    \label {definition:descending chain condition}
    对称的, 一个集合族称满足降链条件 (descending chain condition) 当且仅当其任意降链都起于某个元素.
\end{definition}

\begin{definition}[Noetherian]
    \setlabel {Noetherian}
    \label {definition:noetherian topological space}
    一个拓扑空间称 Noetherian 当且仅当其开集族满足 \ref{definition:ascending chain condition}.
\end{definition}

\begin{lemma}
    满足 \ref{definition:descending chain condition} 则其为良基集.
\end{lemma}

\begin{corollary}
    \ref{definition:noetherian topological space} 空间的闭集族满足 \ref{definition:descending chain condition}.
\end{corollary}

\begin{lemma}
    \ref{definition:noetherian topological space} 空间的任意子空间仍 \ref{definition:noetherian topological space}. 
\end{lemma}

\begin{lemma}
    有限个 \ref{definition:noetherian topological space} 空间的并仍 \ref{definition:noetherian topological space}.
\end{lemma}

\begin{lemma}
    \ref{definition:noetherian topological space} 空间都 \ref{definition:compact topological space},
    如果任意开集都 \ref{definition:compact topological space}, 则 \ref{definition:noetherian topological space}.

    \begin{proof}
        给出开覆盖 \(U_\alpha\), 假使无有限子覆盖, 则可以构造升链 \(\bigcup_{i = 1}^n U_{\alpha_i}\), 
        使得 \(U_{\alpha_i} \nsubseteq \bigcup_{j = 1}^{i-1} U_{\alpha_j}\), 于是有矛盾.

        反之, 如果所有开集都 \ref{definition:compact topological space}, 对任意升链 \(\{U_i\}\), 
        \(\bigcup_{i = 1}^n U_i\) 有有限子覆盖, 于是升链止于某个元素.
    \end{proof}
\end{lemma}

\begin{lemma}
    \ref{definition:noetherian topological space} 空间有只有限个 \ref{definition:irreducible component}.

    \begin{proof}
        依赖良基给出的归纳原理, 只需证明任何真闭子集都有有限个 \ref{definition:irreducible component}, 则其有有限个 \ref{definition:irreducible component}.
        如若该闭子集 \ref{definition:irreducible component}, 则其本身就为唯一的 \ref{definition:irreducible component}, 否则其为两个真闭子集的并, 于是有有限个 \ref{definition:irreducible component}.
    \end{proof}
\end{lemma}

\begin{lemma}
    连续满射保 \ref{definition:noetherian topological space}, 亦保局部 \ref{definition:noetherian topological space}.
\end{lemma}

\begin{definition}[Krull 维度]
    \setlabel {Krull 维度}
    \label {definition:Krull dimension}
    对拓扑空间 \(X\), 定义其 Krull 维度为使得下由 \ref{definition:irreducible topological space} 闭集链 \(Z_i\) 存在的最大 \(\dim (X) := n \in \mathbb{Z}_{\geq 0} \cup \{\infty, - \infty\}\), 可见 \ref{definition:length of a module lattice}.

    \[
        Z_0 \subsetneq Z_1 \subsetneq \cdots \subsetneq Z_n \subseteq X
    \]
\end{definition}

\begin{definition}
    一个点 \(x\) 的 Krull 维度为包含该点的开集的 \ref{definition:Krull dimension} 的极小值,
    记作 \(\dim_X (x)\).
\end{definition}

\begin{example}
    \(\dim (\mathbb{R}^n) = 0\).
\end{example}

\begin{example}
    \ref{example:sierpinski two point set} 的 Krull 维度为 \(1\).
\end{example}

\begin{definition}[等维]
    \setlabel {等维}
    \label {definition:equidimensional topological space}
    一个空间称等维 (equidimensional) 当且仅当其所有不可约分支的 Krull 维度相同.
\end{definition}

\begin{definition}[余维度]
    \setlabel {余维度}
    \label {definition:codimension}
    对于闭集 \(Z \subseteq X\), 定义其余维度 (codimension) 为极大的 \(\mathrm{codim} (Z,X) := e \in \mathbb{Z}_{\geq 0} \cup \{\infty\}\) 
    使得以下由 \ref{definition:irreducible topological space} 闭集构成的序列存在.

    \[
        Z = Z_0 \subsetneq Z_1 \subsetneq Z_2 \subsetneq \cdots \subsetneq Z_e \subseteq X
    \]
\end{definition}

\begin{lemma}
    \label {lemma:codimension in open sets}
    给出开集 \(U \subseteq X\), 有
    \[
        \mathrm{codim} (Y,X) = \mathrm{codim} (Y \cap U, U)
    \]

    \begin{proof}
        注意映射 \(T \mapsto \overline{T}\) 与 \(T \mapsto T \cap U\) 给出链的双射, 因为 \ref{definition:irreducible topological space} 空间不能被 \(X \setminus U\) 约去.
    \end{proof}
\end{lemma}

\begin{definition}[悬链空间]
    \setlabel {悬链}
    \label {definition:catenary topological space}
    一个空间称悬链 (catenary) 当且仅当对于 \ref{definition:irreducible component} 闭集 \(T \subseteq T^\prime\), 存在 \(\mathrm{codim} (T, T^\prime)\),
    且任意极大链等长.

    \[
        T = T_0 \subsetneq T_1 \subsetneq T_2 \subsetneq \cdots \subsetneq T_e \subseteq T^\prime
    \]
\end{definition}

\begin{lemma}
    \(X\) \ref{definition:catenary topological space} 当且仅当其有 \ref{definition:catenary topological space} 的开覆盖.

    \begin{proof}
        知对于某个开集 \(U\), 有 \(T \cap U = \varnothing\), 于是给出双射 \(T \mapsto \overline{T}\),
        \(T \mapsto T \cap U\) 与 \ref{definition:catenary topological space} 的链的双射.
    \end{proof}
\end{lemma}

\begin{lemma}
    \ref{definition:catenary topological space} 当且仅当 \(\infty \notin \mathrm{ran} (\mathrm{codim})\) 且
    对于任意 \ref{definition:irreducible topological space} 闭集 \(A \subseteq B \subseteq C\), 有 \(\mathrm{codim} (A,C) = \mathrm{codim} (A,B) + \mathrm{codim} (B,C)\).

    \begin{proof}
        相邻两项 \(\mathrm{codim}\) 为 \(1\).
    \end{proof}
\end{lemma}

\subsubsection{可构造性}

\begin{definition}[保紧]
    \setlabel {保紧}
    \label {definition:compact function}
    一个连续函数 \(f : X \to Y\) 称保紧 (compact) 当且仅当对于任意紧集 \(K \subseteq Y\), \(f^{-1} (K)\) 紧.
\end{definition}

\begin{definition}[退紧]
    \setlabel {退紧}
    \label {definition:retrocompact}
    一个子集 \(A \subseteq X\) 称退紧 (retrocompact) 当且仅当嵌入 \(A \to X\) \ref{definition:compact function}.
\end{definition}

\begin{definition}[可构造]
    \setlabel {可构造}
    \label {definition:constructible topology space}
    一个子集 \(A \subseteq X\) 称可构造 (constructible) 当且仅当 \(A = \bigcup U_\alpha \cap (X \setminus V_\alpha)\),
    其中 \(U_\alpha,V_\alpha\) 为开, \ref{definition:retrocompact} 子集.
\end{definition}

\begin{definition}[局部可构造]
    \setlabel {局部可构造}
    \label {definition:locally constructible}
    子集 \(A \subseteq X\) 称局部可构造 (locally constructible) 当且仅当存在 \(X\) 的开覆盖 \(\{U_\alpha\}\),
    使得 \(A \cap U_\alpha\) 可构造.
\end{definition}

\subsubsection{Jacobson 空间}

\begin{definition}[Jacobson 空间]
    \setlabel {Jacobson}
    \label {definition:Jacobson space}
    若空间 \(X\) 有闭点集合 \(X_0\), 若任意闭集 \(Z \subseteq X\), 都有 \(Z = \overline{Z \cap X_0}\), 则称 \(X\) 为 Jacobson 空间 (Jacobson space).
\end{definition}

\begin{lemma}
    如果对一点 \(x\) 有 \(X_0 \cap \overline{\{x\}}\) 闭, 则 \(X\) 为 \ref{definition:Jacobson space}.
\end{lemma}

\begin{lemma}
    \(X\) 有开覆盖 \(\bigcup U_\alpha\), 则 \(X\) 为 \ref{definition:Jacobson space} 当且仅当每个 \(U_\alpha\) 为 \ref{definition:Jacobson space}.

    另外的 \(X_0 = \bigcup U_{\alpha,0}\).
\end{lemma}

\begin{lemma}
    \ref{definition:Jacobson space} 空间的开子空间仍为 \ref{definition:Jacobson space},
    闭子空间仍为 \ref{definition:Jacobson space}, \ref{definition:constructible topology space} 子集仍为 \ref{definition:Jacobson space}.
\end{lemma}

\ref{definition:Jacobson space} 空间 \(X\) 的性质完全被 \(X_0\) 决定.

\begin{lemma}
    \ref{definition:Jacobson space} 空间 \(X\) 的闭子空间 \(Y\) 与 \(X_0\) 的闭子空间 \(X_0 \cap Y\) 一一对应.
\end{lemma}

\begin{lemma}
    \ref{definition:Jacobson space} 空间 \(X\) 的 \ref{definition:constructible topology space} 子集 \(A\) 与 \(X_0\) 的 \ref{definition:constructible topology space} 子集 \(A \cap X_0\) 一一对应.
\end{lemma}
