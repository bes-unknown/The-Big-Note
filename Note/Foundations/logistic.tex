\section{逻辑}

断言 (statement) 是一类有明确真值 (T/F) 的语句, 我们定义如下运算:

\begin{definition}
    对于断言 \(A\) 定义其否定 (negation) 为 \(\neg A\), 满足如下真值表.
    \begin{center}
        \begin{tabular}{|c|c|}
            \hline
            \(A\) & \(\neg A\) \\
            \hline
            T & F \\
            \hline
            F & T \\
            \hline
        \end{tabular}
    \end{center}

    对于断言 \(A, B\) 定义其合取 (conjuction) 为 \(A \land B\), 其析取 (disjuction) 为 \(A \lor B\), 满足如下真值表.

    \begin{center}
        \begin{tabular}{|c|c|c|c|c|}
            \hline
            \(A\) & \(B\) & \(A \land B\) & \(A \lor B\) \\
            \hline
            T & T & T & T \\
            \hline
            T & F & F & T \\
            \hline
            F & T & F & T \\
            \hline
            F & F & F & F \\
            \hline
        \end{tabular}
    \end{center}
\end{definition}

\begin{definition}
    如果 \(E(x)\) 在 \(x\) 是某些对象时为断言, 则称 \(E(x)\) 为一个性质 (property).
\end{definition}

我们承认类 (class) 的概念如下, 若对象 \(x\) 属于类 \(A\), 则记为 \(x \in A\), 否则记为 \(x \notin A\).
我们用 \(\{x \in X; E(x)\}\) 表示 \(X\) 中所有满足性质 \(E\) 的对象组成的类.

\begin{definition}
    我们用 \(\exists\) 代表存在某个对象, 用 \(\forall\) 代表所有对象都满足某个性质.

    我们也用 \(\exists !\) 代表存在唯一一个对象.

    我们用 \(a := b\) 标记 \(a\) 由 \(b\) 定义, \(a = b\) 代表 \(a, b\) 仅仅是相同对象的两个表示 (集合论下会重新定义等于).
\end{definition}

我们有如下自明的公式.

\begin{theorem}
    \begin{align}
        \neg \neg A &:= \neg (\neg A) = A \\
        \neg (A \land B) &= \neg A \lor \neg B \\
        \neg (A \lor B) &= \neg A \land \neg B \\
        \neg (\forall x \in X : E(x)) &= \exists x \in X : \neg E(x) \\
        \neg (\exists x \in X : E(x)) &= \forall x \in X : \neg E(x) \\
        \neg (\forall x \in X : (\exists y \in Y : E(x, y))) &= \exists x \in X : (\forall y \in Y : \neg E(x, y)) \\
        \neg (\exists x \in X : (\forall y \in Y : E(x, y))) &= \forall x \in X : (\exists y \in Y : \neg E(x, y))
    \end{align}
\end{theorem}

\begin{definition}
    对于断言 \(A, B\), 我们记断言 \(A\) 能推出 \(B\) 为 \(A \implies B\), 代表
    \begin{equation}
        A \implies B := (\neg A) \lor B
    \end{equation}

    而断言 \(A, B\) 等价 (\(A \iff B\)) 意味着 \((A \implies B) \land (B \implies A)\).
\end{definition}

\begin{theorem}
    我们可以验证下述断言:
    \begin{align}
        (A \implies B) &\iff (\neg B \implies \neg A) \\
        (A \implies B) \land (B \implies C) &\implies (A \implies C)
    \end{align}
\end{theorem}
